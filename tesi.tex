%%%%%%%%%%%%%%%%%%%%%%%%%%%%%%%%%%%%%%%%%12pt: grandezza carattere
                                        %a4paper: formato a4
                                        %openright: apre i capitoli a destra
                                        %twoside: serve per fare un
                                        %   documento fronteretro
                                        %report: stile tesi (oppure book)
\documentclass[12pt,a4paper,openright,twoside]{report}
%
%%%%%%%%%%%%%%%%%%%%%%%%%%%%%%%%%%%%%%%%%libreria per scrivere in italiano
\usepackage[italian]{babel}
%
%%%%%%%%%%%%%%%%%%%%%%%%%%%%%%%%%%%%%%%%%libreria per accettare i caratteri
                                        %   digitati da tastiera come � �
                                        %   si pu� usare anche
                                        %   \usepackage[T1]{fontenc}
                                        %   per� con questa libreria
                                        %   il tempo di compilazione
                                        %   aumenta
\usepackage[latin1]{inputenc}
%
%%%%%%%%%%%%%%%%%%%%%%%%%%%%%%%%%%%%%%%%%libreria per impostare il documento
\usepackage{fancyhdr}
%
%%%%%%%%%%%%%%%%%%%%%%%%%%%%%%%%%%%%%%%%%libreria per avere l'indentazione
%%%%%%%%%%%%%%%%%%%%%%%%%%%%%%%%%%%%%%%%%   all'inizio dei capitoli, ...
\usepackage{indentfirst}
%
%%%%%%%%%libreria per mostrare le etichette
%\usepackage{showkeys}
%
%%%%%%%%%%%%%%%%%%%%%%%%%%%%%%%%%%%%%%%%%libreria per inserire grafici
\usepackage{graphicx}
%
%%%%%%%%%%%%%%%%%%%%%%%%%%%%%%%%%%%%%%%%%libreria per utilizzare font
                                        %   particolari ad esempio
                                        %   \textsc{}
\usepackage{newlfont}
%
%%%%%%%%%%%%%%%%%%%%%%%%%%%%%%%%%%%%%%%%%librerie matematiche
\usepackage{amssymb}
\usepackage{amsmath}
\usepackage{latexsym}
\usepackage{amsthm}

\usepackage{tabularx}

\usepackage[dvipsnames]{xcolor}

\usepackage{listings}
\usepackage{color}

\definecolor{dkgreen}{rgb}{0,0.6,0}
\definecolor{gray}{rgb}{0.5,0.5,0.5}
\definecolor{mauve}{rgb}{0.58,0,0.82}

\lstset{frame=tb,
  language=Java,
  aboveskip=3mm,
  belowskip=3mm,
  showstringspaces=false,
  columns=flexible,
  basicstyle={\small\ttfamily},
  numbers=none,
  numberstyle=\tiny\color{gray},
  keywordstyle=\color{blue},
  commentstyle=\color{dkgreen},
  stringstyle=\color{mauve},
  breaklines=true,
  breakatwhitespace=true,
  tabsize=3
}

\lstdefinelanguage{JavaScript}{
  keywords={typeof, new, true, false, catch, function, return, null, catch, switch, var, if, in, while, do, else, case, break},
  keywordstyle=\color{blue}\bfseries,
  ndkeywords={class, export, boolean, throw, implements, import, this},
  ndkeywordstyle=\color{darkgray}\bfseries,
  identifierstyle=\color{black},
  sensitive=false,
  comment=[l]{//},
  morecomment=[s]{/*}{*/},
  commentstyle=\color{purple}\ttfamily,
  stringstyle=\color{red}\ttfamily,
  morestring=[b]',
  morestring=[b]"
}

\lstdefinestyle{json}
{
  showstringspaces    = false,
  keywords            = {false,true},
  alsoletter          = 0123456789.,
  morestring          = [s]{"}{"},
  stringstyle         = \ifcolonfoundonthisline\JSONstringvaluestyle\fi,
  MoreSelectCharTable =%
    \lst@DefSaveDef{`:}\colon@json{\processColon@json},
  basicstyle          = \ttfamily,
  keywordstyle        = \ttfamily\bfseries,
}

\lstdefinelanguage{Kotlin}{
  keywords={package, as, typealias, this, super, val, var, fun, for, null, true, false, is, in, throw, return, break, continue, object, if, try, else, while, do, when, yield, typeof, yield, typeof, class, interface, enum, object, override, public, private, get, set, import, abstract, },
  keywordstyle=\color{NavyBlue}\bfseries,
  ndkeywords={@Deprecated, Iterable, Int, Integer, Float, Double, String, Runnable, dynamic},
  ndkeywordstyle=\color{BurntOrange}\bfseries,
  emph={println, return@, forEach,},
  emphstyle={\color{OrangeRed}},
  identifierstyle=\color{black},
  sensitive=true,
  commentstyle=\color{gray}\ttfamily,
  comment=[l]{//},
  morecomment=[s]{/*}{*/},
  stringstyle=\color{ForestGreen}\ttfamily,
  morestring=[b]",
  morestring=[s]{"""*}{*"""},
}



\oddsidemargin=30pt \evensidemargin=20pt%impostano i margini
\hyphenation{sil-la-ba-zio-ne pa-ren-te-si}%serve per la sillabazione: tra parentesi
					   %vanno inserite come nell'esempio le parole
%					   %che latex non riesce a tagliare nel modo giusto andando a capo.

%
%%%%%%%%%%%%%%%%%%%%%%%%%%%%%%%%%%%%%%%%%comandi per l'impostazione
                                        %   della pagina, vedi il manuale
                                        %   della libreria fancyhdr
                                        %   per ulteriori delucidazioni
\pagestyle{fancy}\addtolength{\headwidth}{20pt}
\renewcommand{\chaptermark}[1]{\markboth{\thechapter.\ #1}{}}
\renewcommand{\sectionmark}[1]{\markright{\thesection.\ #1}{}}
\rhead[\fancyplain{}{\bfseries\leftmark}]{\fancyplain{}{\bfseries\thepage}}
\cfoot{}
%%%%%%%%%%%%%%%%%%%%%%%%%%%%%%%%%%%%%%%%%
\linespread{1.3}                        %comando per impostare l'interlinea
%%%%%%%%%%%%%%%%%%%%%%%%%%%%%%%%%%%%%%%%%definisce nuovi comandi
%
\begin{document}

\begin{titlepage}
\begin{center}
{{\Large{\textsc{Alma Mater Studiorum $\cdot$ Universit\`a di
Bologna}}}} \rule[0.1cm]{13.8cm}{0.1mm}
\rule[0.5cm]{13.8cm}{0.6mm}
{\bf SCUOLA DI SCIENZE\\
Corso di Laurea in Informatica }
\end{center}
\vspace{25mm}
\begin{center}
 {\large{\bf UTILIZZO DI KOTLIN PER SVILUPPO }}\\
 \vspace{2mm}
 {\large{\bf DI APPLICAZIONI MOBILI: }}\\
 \vspace{2mm}
 {\large{\bf UN CASO DI STUDIO}}\\
 \vspace{2mm}

 \end{center}
 \vspace{30mm}
 \par
 \noindent
 \begin{minipage}[t]{0.5\textwidth}
 {\large{\bf Relatore:\\
 Dott.\\
 LUCA BEDOGNI}}\\
 \end{minipage}
 \hfill
 \begin{minipage}[t]{0.5\textwidth}\raggedleft
 {\large{\bf Presentata da:\\
 ALESSIO KOCI}}
\end{minipage}
\vspace{30mm}
\begin{center}
{\large{\bf Sessione III\\%inserire il numero della sessione in cui ci si laurea
Anno Accademico 2016/17}}%inserire l'anno accademico a cui si è iscritti
\end{center}
\end{titlepage}

\include{Dedica}
\tableofcontents                        %crea l'indice
%%%%%%%%%%%%%%%%%%%%%%%%%%%%%%%%%%%%%%%%%imposta l'intestazione di pagina
\rhead[\fancyplain{}{\bfseries\leftmark}]{\fancyplain{}{\bfseries\thepage}}
\lhead[\fancyplain{}{\bfseries\thepage}]{\fancyplain{}{\bfseries
INDICE}}
%%%%%%%%%%%%%%%%%%%%%%%%%%%%%%%%%%%%%%%%%non numera l'ultima pagina sinistra
\clearpage{\pagestyle{empty}\cleardoublepage}
\listoffigures                          %crea l'elenco delle figure
%%%%%%%%%%%%%%%%%%%%%%%%%%%%%%%%%%%%%%%%%non numera l'ultima pagina sinistra
\clearpage{\pagestyle{empty}\cleardoublepage}
\listoftables                           %crea l'elenco delle tabelle
%%%%%%%%%%%%%%%%%%%%%%%%%%%%%%%%%%%%%%%%%non numera l'ultima pagina sinistra
\clearpage{\pagestyle{empty}\cleardoublepage}

\chapter*{Introduzione}                 %crea l'introduzione (un capitolo
                                        %   non numerato)

%%%%%%%%%%%%%%%%%%%%%%%%%%%%%%%%%%%%%%%%%imposta l'intestazione di pagina
\rhead[\fancyplain{}{\bfseries
Introduzione}]{\fancyplain{}{\bfseries\thepage}}
\lhead[\fancyplain{}{\bfseries\thepage}]{\fancyplain{}{\bfseries
INTRODUZIONE}}
%%%%%%%%%%%%%%%%%%%%%%%%%%%%%%%%%%%%%%%%%aggiunge la voce Introduzione
                                        %   nell'indice
\addcontentsline{toc}{chapter}{Introduzione}
L'idea di realizzare un'applicazione gestionale su piattaforma Android, è nata dall'esigenza di risolvere alcune problematiche reali riscontrate durante l'organizzazione di eventi, spese e commissioni da svolgere fra gruppi di persone.\\
Lo sviluppo iniziale si avvaleva di Java come linguaggio di programmazione, successivamente con l'annuncio ufficiale a Maggio 2017 del supporto di Google a Kotlin, come nuovo linguaggio di programmazione per Android, l'applicazione è stata riscritta completamente, utilizzando Kotlin.\\
Lo sviluppo dell'applicazione in Kotlin ha permesso di  analizzare le funzionalità del nuovo linguaggio sia teoricamente che con un'implementazione pratica che ha rendendo possibile anche un confronto diretto con Java.\\
In questa tesi verranno illustrate le principali funzionalità e caratteristiche del linguaggio Kotlin, e del BaaS Firebase utilizzato per la gestione della parte server. Dal terzo capitolo invece verrà presa in considerazione l'applicazione realizzata per lo studio di Kotlin, mostrando un architettura ad alto livello dell'infrastruttura dei servizi server e delle funzionalità dell'applicazione. Successivamente verranno analizzate porzioni di codice rilevani e discusse le implementazioni sviluppate per la realizzazione dell'applicazione. Infine nell'ultimo capitolo viene descritto l'attuale stato dell'arte, il confronto fra Java e Kotlin ed eventuali sviluppi futuri.
%%%%%%%%%%%%%%%%%%%%%%%%%%%%%%%%%%%%%%%%%non numera l'ultima pagina sinistra
\clearpage{\pagestyle{empty}\cleardoublepage}


\chapter{Kotlin}                %crea il capitolo
%%%%%%%%%%%%%%%%%%%%%%%%%%%%%%%%%%%%%%%%%imposta l'intestazione di pagina
\lhead[\fancyplain{}{\bfseries\thepage}]{\fancyplain{}{\bfseries\rightmark}}
\pagenumbering{arabic}                  %mette i numeri arabi
\section{Storia}                 %crea la sezione

Kotlin \'e un linguaggio di programmazione open source\footnote{https://github.com/JetBrains/kotlin}, basato sulla JVM (Java Virtual Machine), con la caratteristica di essere orientato agli oggetti e staticamente tipizzato. \\
Lo sviluppo di Kotlin \'e iniziato nel 2010 dall'azienda JetBrains, conosciuta nel modo dello sviluppo software Java per la realizzazione di diversi IDE (Integrated development environment), tra cui: Intellij IDEA, sul quale si basa l'attuale IDE di Google dedicato alla programmazione Android, chiamato: Android Studio.\\
Il team di JetBrains, scelse di iniziare a realizzare un nuovo linguaggio di programmazione, per risolvere problematiche riscontrate durante la realizzazione dei loro IDE, in particolare Dmitry Jemerov, un programmatore di JetBrains e sviluppatore di Kotlin, rivel\'o durante un'intervista\footnote{https://www.infoworld.com/article/2622405/java/jetbrains-readies-jvm-based-language.html} che non esisteva ancora nessun linguaggio con le potenzialit\'a e facilit\'a d'uso richieste dal team di JetBrains, ad eccezione del linguaggio Scala che offriva grossi vantaggi nello sviluppo, ma aveva un tempo di compilazione molto lento, JetBrains scelse quindi di realizzare il linguaggio Kotlin basandosi sulla JVM, in seguito nel 2012 il progetto Kotlin venne resto Open Source sotto licenza "Apache 2 license".\\
Java fin dalle sue prime versioni, \'e sempre stato uno dei linguaggi  pi\'u usati e conosciuti ma presenta diverse imperfezioni e problemi che spinse il team di JetBrains a iniziare lo sviluppo di un suo linguaggio, con una sintassi semplice che prendesse in considerazione alcuni spunti e idee introdotte da linguaggi come CSharp, Scala, Groovy, ECMAScript, Go, Python, ma continuasse a basarsi sulla JVM.\\
L'idea di utilizzare pattern e idee di linguaggi preesistenti permise agli sviluppatori di introdurre la facilit\'a sintattica, potenzialit\'a e caratteristiche testate a lungo da altri linguaggi, senza dover apportare grosse innovazioni, rendendo quindi il linguaggio leggibile e comprensibile anche da chi non lo conoscesse.\\
Lo sviluppo di Kotlin si basava principalmente sul miglioraramento di Java, ma dato che anche Android utilizzava la JVM, gli sviluppatori cercarono di adattarlo e renderlo ottimizzato anche per Android, come Java.\\
La comunit\'a di sviluppatori che cominci\'o a utilizzare Kotlin crebbe enormemente, tra il 2016 e il 2017, in solo un anno, le linee di codice scritte in Kotlin su progetti presenti su GitHub quadruplicarono passando da 2.4 milioni a 10 milioni\footnote{https://blog.jetbrains.com/kotlin/2017/03/kotlin-1-1/} \\

\begin{figure}[!hb]
  \centering
  \includegraphics[width=0.85\textwidth]{immagini/kotlin_grafico_incremento.png}
  \caption{Grafico delle linee di codice scritte in Kotlin su Github.}\label{fig:Grafico delle linee di codice scritte in Kotlin su Github}
\end{figure}

Nel 2015 Google prese in considerazione l'utilizzo di Kotlin come plugin per Android Studio, e dopo vari test nel 2017 durante la conferenza Google IO 2017, arriv\'o l'annuncio che ufficializzava\footnote{https://android-developers.googleblog.com/2017/05/android-announces-support-for-kotlin.html} Kotlin come nuovo linguaggio di programmazione per lo sviluppo di applicazioni Android, senza escludere e rinunciare a Java, su cui si basa l'SDK di Android.\\
Gli stessi sviluppatori Android, dopo aver testato le potenzialit\'a di Kotlin ne rimasero molto soddisfatti per la praticit\'a, la stabilit\'a e i suoi benefici sintattici e funzionali, Kotlin \'e infatti un linguaggio molto coinciso, espessivo, strutturato sulla tipizzazione che mette a disposizione costrutti per evitare errori a puntatori nulli.\\
Il supporto completo di Kotlin su Android venne garantito attraverso una buona integrazione con Android Studio 3.0 e un plugin Kotlin per le versioni precedenti delle IDE, qualsiasi progetto che utilizzava Java poteva essere parzialmente o completamente convertito in Kotlin.





\section{Caratteristiche}
Il linguaggio Kotlin \'e stato sviluppato in ambito aziendale e non accademico, come spesso accade per altri linguaggi, rimanendo in fase beta  per 7 anni, periodo in cui gli stessi programmatori che lavoravano presso JetBrains ne testarono le funzionalit\'a, fino a raggiungere nel 2017 la prima versione stabile: la 1.0. \\
Kotlin \'e nato quindi all'interno di un team di sviluppatori che dopo anni di esperienza acquisita con Java e altri linguaggi hanno realizzato un linguaggio mirato a risolvere problematiche concrete riscontrati dagli stessi sviluppatori.\\
Prendendo spunto dalle problematiche di Java e da buone regole introdotte da alcuni linguaggi imperativi e funzionali, Kotlin \'e stato modellato in modo tale da aggiungere funzionalit\'a utili sia a livello sintattico che a livello prestazionale, offrendo quindi al programmatore strumenti, caratteristiche e implementazioni semplici e utili ma molto potenti.\\
Un altro aspetto importante su cui il team di JetBrains ha prestato molta attenzione \'e stata la buona integrazione del suo plugin con Android Studio. Il supporto dato dal plugin \'e tale da supportare il programmatore in ogni momento, proponendo la riscrittura di porzioni di codice per renderlo pi\'u coinciso, allertare il programmatore in caso di possibili puntatori nulli, offrire una conversione automatica del codice Java in Kotlin e ove possibile cerchere di avvisare lo sviluppatore su possibili problemi di prestazione ed errori sintattici.\\
Le caratteristiche pi\'u importanti offerte da Kotlin sono l'interoperabilit\'a con Java, permettendo l'utilizzo di librerie Java e Kotlin simultaneamente, l'introduzione di alcune caratteristiche dei linguaggi di ordine superiore, la tipizzazione statica delle variabili, l'inferenza di tipo e soprattutto il null-safety consentendo di differenziare il tipo nullabile e il tipo non-nullabile, prevenendo quindi errori di "NullPointerExpetions".\\
Il codice prodotto in Kotlin \`e inoltre pi\'u compatto, coninciso e meno verboso grazie alle dataclass, il supporto delle lambda function e altri costrutti utili.


\subsection{Interoperabilit\'a}
I linguaggi Kotlin e Java sono fortemente intercompatibili, permettendo quindi a entrambi i linguaggi di coesistere all'interno dello stesso progetto e di richimare funzioni e parti di codice in Java da Kotlin e viceversa, poich\'e entrambi i linguaggi producono Java Bytecode\footnote{http://kotlinlang.org/docs/reference/java-interop.html}.\\
Prendiamo in considerazione il classico esempio ``Hello word'' scritto in Kotlin e in Java:

\begin{lstlisting}[language=java,caption={Hello World in Kotlin}]
fun main(args : Array<String>) {
  println("Hello, world!")
}
\end{lstlisting}

\begin{lstlisting}[language=java,caption={Hello World in Java}]
public class Hello {
    public static void main(String[] args) {
        System.out.println("Hello, world!");
    }
}
\end{lstlisting}

Il compilatore di Kotlin prendendo come input il file ``Hello.kt'' produrr\'a un JAR eseguibile da Java ``Hello.jar''

\begin{lstlisting}[language=bash,caption={Compilazione di un programma Kotlin}]
$ kotlinc Hello.kt -include-runtime -d Hello.jar
$ java -jar Hello.jar
$ Hello, world!
\end{lstlisting}

\'E possibile chiamare classi Java all'interno di funzioni Kotlin e viceversa:


\begin{lstlisting}[language=Kotlin,caption={Chiamare Java da Kotlin}]
 class KotlinClass {
     fun kotlinDoSomething() {
         val javaClass = JavaClass()
         javaClass.javaDoSomething()
         println(JavaClass().prop)
     }
 }
 \end{lstlisting}

 \begin{lstlisting}[language=java,caption={Chiamare Kotlin da Java}]
 public class JavaClass {
     public String getProp() { return "Hello"; }
     public void javaDoSomething() {
         new KotlinClass().kotlinDoSomething();
     }
 }
 \end{lstlisting}

%https://developer.android.com/kotlin/index.html
\subsection{Performance}
I tempi di compilazione e d'esecuzione di un programma scritto in Kotlin sono molto simili a Java poich\'e entrambi producono bytecode per la JVM.\\
Nei progetti Android, utilizzare la libreria Kotlin non comporta un grande aumento nella dimensione dell'APK, Kotlin introduce circa 7000 metodi aggiuntivi a run-time che corrispondono ad un aumento di 1MB nell'APK finale.\\
L'impatto di questa libreria aggiuntiva, anche se aumenta la dimensione dell'APK, porta tanti vantaggi poich\'e grazie alle nuove caratteristiche introdotte da Kotlin, non sar\'a necessario utilizzare librerie esterne come: RxJava, Guava, ButterKnife che spesso vengono importate in progetti Android aumentando considerevolmente la dimensione finale dell'APK.\\
In termini di performance Kotlin pone alcuni miglioramenti prestazionali nelle funzioni di ordine superiore e lambda function, dimostrandosi pi\'u ottimizzato e veloce nei confronti di Java, che ha introdotto queste nuove funzionalit\'a solo dalla versione 8.\\
Altri miglioramenti di performance si possono notare nella memorizzazione delle variabili, poich\'e Kotlin utilizza una buona gestione dell'Autoboxing, permettendo di instanziare un oggetto solo quando \'e strettametne necessario, altri miglioramenti riguardano, le inline-function e le funzioni di ordine superiore che risultano pi\'u veloci rispetto a Java, poich\'e Kotlin supporta le lambda function nativamente a differenza di Java che per supportarle crea oggetti o utilizza chiamate virtuali.\\
Kotlin quindi non comporta alcun svantaggio al programmatore oltre all'aumento della dimensione finale dell'APK corrispondente a qualche MB aggiuntivo, di conseguenza pur non aggiungendo sostanziosi miglioramenti in termini di performance rispetto a Java, offre numerosi vantaggi sintattici.

\newpage

%https://sites.google.com/a/athaydes.com/renato-athaydes/posts/kotlinshiddencosts-benchmarks



\subsection{Coroutines}
Dalla versione 1.1 Kotlin, introduce in fase sperimentale le Coroutines, consentendo agli sviluppatore di testarne le funzionali\'a, semplicemente aggiungendo nel file di configurazione ``build.gradle'', presente all'interno dell sorgente del progetto Android, un eccezione:
\begin{lstlisting}[language=bash,caption={Gradle Coroutines }]
kotlin {
 experimental {
  coroutines 'enable'
 }
}
\end{lstlisting}

Le Coroutines offrono un modo per scrivere sequenzialmente, programmi che operano in maniera asincrona, la differenza sostanziale rispetto a Java e altri linguaggi \'e il modo e l'ordine con cui vengono scritte la parti di codice asincrone.\\
Le coroutines permettono di scrivere istruzioni, una dopo l'altra, con la possibilit\'a di sospendere l'esecuzione e attendere momentaneamente che un risultato sia disponibile e successivamente riprendere l'esecuzione, aumentano la facilit\'a di lettura del codice e migliorando l'utilizzo della memoria a differenza dei thread.\\
Le operazioni che utilizzano i Thread spesso riguardano la gestione di processi che operano in rete (network I/O), per leggere file locali o che sfruttano i Thread per un fare dei calcoli che fanno un intensivo della CPU e GPU, bloccando l'utilizzo del dispositivo fino alla loro terminazione.\\
La soluzione offerta da Java \'e quella tradizionale, che consiste di creare un Thread che opera in background, ma in termini di prestazioni \'e svantaggioso poich\'e creare e gestire molti Thread \'e un operazione costosa e complessa.\\
Attraverso le Coroutines di Kotlin invece \'e possibile sospendere funzioni che possono interrompere l'eseguzione del programma principale e riprenderle successivamente, queste funzioni vengono chiamate ``funzioni di sospensione'' e sono contrassegnate con la keyword ``suspend''.\\
Queste funzioni di sospensione, sono normali funzioni con parametri e valori di ritorno che permettono di sospendere una coroutine, il vantaggio \'e che la sospensione e la ripresa di queste funzioni \'e ottimizzata per avere un costo quasi nullo, inoltre la libreria pu\'o dedicere di proseguire l'eseguizione senza la sospensione, se il risultato \'e gi\'a disponibile.

\begin{lstlisting}[language=kotlin,caption={Esempio Kotlin Coroutines }]
  suspend fun doSomething(foo: Foo): Bar {
      ...
  }
  fun <T> async(block: suspend () -> T)
  async {
      ...
      doSomething(foo)
      ...
  }

\end{lstlisting}

Async \'e una normale funzione (non di sospensione) che contiene un una funzione di sospenzione all'interno.\\
Le coroutine sono completamente implementate attraverso una tecnica di compilazione (nessun supporto da parte della JVM o del sistema operativo), fondamentalmente, ogni funzione di sospensione viene trasformata in una macchina di stato, dove gli stati corrispondono a sospendere le chiamate. Subito prima della sospensione, lo stato successivo viene memorizzato in un campo di una classe generata dal compilatore insieme alle variabili locali. Alla ripresa di quella routine, le variabili locali vengono ripristinate e la macchina procede dallo stato successivo a quello della sospensione.\\
Molti meccanismi asincroni disponibili in altri linguaggi possono essere implementati con Kotlin utilizzano le coroutines, come ad esempio async/await di CSharp, ``channels and select'' di Go e ``generators/yeald'' di Python.


\section{Variabili}
Kotlin utilizza due keyword differenti per dichiarare una variabile: "var" e "val" seguite dal nome che si vuole assegnare alla variabile e il tipo opzionale della variabile (nel caso non fosse definito il tipo, Kotlin attraverso il Type Inference, riconosce automaticamente il tipo della variabile in base al tipo del suo primo assegnamento).

\begin{itemize}                         %crea un elenco puntato
\item \textbf{Var}: Variabile mutabile, permette ad una variabile di modificare il suo valore con un riassegnamento durante l'eseguzione del programma o posticiparne l'inizializzazione, indicando solamente il tipo della variabile
\item \textbf{Val}: Variabile Immutabile, permette di dichiarare una variabile di sola lettura (equivalente a "final" in Java). L'inizializzazione di una variabile val non pu\'o essere posticipata
\end{itemize}


Un'ultima caratteristica introdotta da Kotlin nell'inizializzazione di una nuova variabile sono la ``Lazy  Initialization'' e la ``Late Initialization'', due nuovi modi per inizializzre una variabile.
\begin{itemize}                         %crea un elenco puntato
\item \textbf{Lazy}: consente di delegare ad una funzione l'inizializzazione della variabile, il risultato della funzione verr\'a assegnato alla variabile, in seguito quando verr\'a effettuato l'accesso alla variabile la funzione non sar\'a rieseguita ma verr\'a solamente passato il valore
\item \textbf{Late}: permette di posticipare l'inizializzazione di una variabile, se si tenter\'a di acceddere alla variabile prima che essa venga inizializzata si ricever\'a un errore. Late \'e  stato principalmente introdotto per supportare la ``dependency injection'', ma pu\'o essere comunque utilizzato dal programmatore per scrivere codice efficiente
\end{itemize}

\begin{lstlisting}[language=Kotlin,caption={Esempio Late e Lazy Initialization in Kotlin}]
lateinit var prova: String
val lazyString = lazy { readStringFromDatabase() }
\end{lstlisting}


\subsection{Autoboxing}
Java pone due differenze quando si parla di variabili, mette a disposizione i principali tipi primitivi (int, boolean, byte, long, short, float, double, char) e le loro corrispondenti classi (Int, Boolean, Byte, Long, Short, Float, Double, Char).\\
Uno dei principali cambiamenti introdotti da Kotlin \'e stato quello di rendere accessibile allo sviluppatore tutte le varibili come se fossero oggetti.\\
%https://docs.oracle.com/javase/1.5.0/docs/guide/language/autoboxing.html
La differenza fra i tipi primitivi e gli oggetti risiede nel loro utilizzo, i primi indicano solamente il tipo di una variabile, mentre gli oggetti incapsulano il tipo e ne aggiungono funzionalit\'a e metodi aggiuntivi, inoltre il tipo primitivo non pu\'o assumere valore nullo. \\
Kotlin operando ad alto livello, rimuove e astrae le due distinzioni poich\'e di default quando viene inizializzata una nuova variabile, la identifica come un oggetto, consentendo allo sviluppatore di utilizzare i metodi aggiuntivi ad esso associati e solo in fase di compilazione il compilatore di Kotlin controller\'a se l'oggetto \'e strettamente necessario o pu\'o essere sostituito dal suo corrispondente tipo primivio.

\begin{center}
    \begin{tabular}{ | l | l | l | p{5cm} |}
    \hline
    Tipo & Oggetto & Dimensione \\ \hline
    int & Int & 32 bits\\ \hline
    boolean & Boolean & 1 bits\\ \hline
    byte & Byte & 8 bits\\ \hline
    long & Long & 64 bits\\ \hline
    short & Short & 16 bits\\ \hline
    float & Float & 32 bits\\ \hline
    double & Double & 64 bits\\ \hline
    char & Char & 16 bits\\ \hline

    \end{tabular}
\end{center}
L'unico tipo introdotto da Kotlin \'e ``Nothing'', un tipo senza istanze, molto simile al concetto del tipo ``Any''.\\
Any \'e superclasse di tutti i tipo, Nothing contrariamente \'e la sottoclasse di tutti i tipi.\\
Nothing viene utilizzato dal compilatore per indicare che una funzione non ritorna nessun valore, in particolare viene utilizzato per indicare che \'e presente un loop infinito, oppure per inizializzare una variabile che non contiene nessun elemento, infatti \'e la base per definire le funzioni emptyList(), emptySet(), introdotte da Kotlin.


\subsection{Optional}
Le variabili sono pressocch\'e le stesse che sono presenti in Java, con la particolari\'a che Kotlin cerca di evitare alcuni problemi dovuti a referenze a puntatori nulli (NullPointerException). \\
Kotlin richiede che una variabile a cui assegnamo un valore nullo sia dichiarata con l'operatore ``?'', in caso contrario mostrer\'a un errore in fase di compilazione.

\begin{lstlisting}[language=kotlin,caption={Esempio 1 Safe call operator Kotlin}]
var esempio1: String? = null //corretto
var esempio2: String = null //errore
\end{lstlisting}

Il safe call operator ``?'' serve ad indicare che la variabile pu\'o assumere in qualsiasi momento un valore nullo, e lascia al programmatore la responsabilit\'a e la possibilit\'a di accedervi ugualmente per leggerne il valore, con l'utilizzo dell operatore ``!!''.


\begin{lstlisting}[language=kotlin,caption={Esempio 2 Safe call operator Kotlin}]
val nome = getName()!!
\end{lstlisting}

In altrnativa, attraverso il Smart Casting, l'operatore ``!!'' si pu\'o omettere, poich\'e il compilatore capisce automaticamente che la variabile non potr\'a assumere il valore nullo.

\begin{lstlisting}[language=kotlin,caption={Smart Casting in Kotlin}]
fun getName(): String? {..}
val name = getName()
if (name != null) {
  println(name.length)
}
//forma contratta
println(name?.length)
\end{lstlisting}

\subsection{String Template}
La gestione delle stringhe in Kotlin si differisce dalla gestione di Java per l'aggiunta di nuove caratteristiche, tra cui il ``String Template'' disponibile gi\'a in altri linguaggi.
Lo string Template consiste nel fare riferimento a variabili, durante la rappresentazione di stringhe, aggiungendo il prefisso  ``\textdollar'' al nome della variabile, nel caso si volesse accedere ad una sua propriet\'a \'e necessario utilizzare le parentesi graffe dopo il prefisso. ``\textdollar''.


\begin{lstlisting}[language=kotlin,caption={Esempio String template Kotlin}]
val name = "Sam"
val str = hello $name. Your name has  ${name.length} characters
\end{lstlisting}


\section{Funzioni}
Le funzioni sono definite utilizzando la parola ``fun'' seguite dal nome della funzione, i parametri opzionali e il valore di ritorno anch'esso opzionale.\\
La visibilit\'a di una funzione di default \'e ``public'' ma come in Java pu\'o essere modificata, indicando il tipo di visibilit\'a, seguito dalla definizione della funzione.

\begin{lstlisting}[language=Kotlin,caption={Esempio Funzione Kotlin}]
fun saluta(nome: String): String {
  return "Ciao $nome"
}
\end{lstlisting}

Gli argomenti delle funzioni in Kotlin possono assumere il valore passato dal chiamante della funzione oppure avere un valore di default.\\ Questa caratteristica oltre ad essere utile al programmatore che eviter\'a di inserire controlli all'interno di funzioni o addirittura creare un'altra funzione con parametri diversi, aumenta la leggibilit\'a del codice, rendendolo pi\'u diretto e comprensivo.\\
Un'ultima caratteristica riguardante i parametri delle funzioni \'e la possibilit\'a di indicare l'ordine e il valore a cui assegnare il dato passato per parametro, indicando il nome del parametro della funzione seguito dal simbolo ``=''

\begin{lstlisting}[language=kotlin,caption={Esempio Kotlin Parametri}]
fun buyItem(id:String, status:Boolean = true){...}
buyItem(id=23) // oppure semplicemente: buyItem(23)
\end{lstlisting}

Tutte le funzioni devono restituire un valore, qualora non ci fosse, il valore di default assegnato da Kotlin \'e "Unit" corrispondente a "Void" in Java, l'unica eccezione viene fatta per le "Single expression functions", ovvero funzioni che vengono sritte in una sola linea, solitamente formate da un' unica espressione.
\begin{lstlisting}[language=java,caption={Esempio Single Expression Function in Kotlin}]
fun quadrato(k: Int) = k * k
\end{lstlisting}

\subsection{Funzioni Locali}
Nei linguaggi di programmazione le funzioni sono state introdotte per ridurre la ripetizione di codice gi\'a scritto e migliorarne la sua leggibilit\'a.\\
Il concetto chiave risiede quindi nel creare tante funzioni che eseguono determinate operazioni e restituiscono un valore al chiamante.
Kotlin amplia le funzionalit\'a delle funzioni rendendo disponibili le ``Local Function'' ovvero funzioni che possono essere definite e richiamate all'interno di altre funzioni, con il vantaggio di poter accedere a variabili definite nello scope esterno.

\begin{lstlisting}[language=java,caption={Esempio Funzioni locali}]
  fun printArea(width: Int, height: Int): Unit {
    fun calculateArea(): Int = width * height
    val area = calculateArea()
    println("The area is $area")
  }
\end{lstlisting}

\subsection{Funzioni di ordine superiore }
Le funzioni di ordine superiori sono funzioni che possono accettare come argomento una funzione stessa, o restituirne una.\\
Queste funzioni sono utilizzate molto nella programmazione funzionale ma sono state introdotte anche nei linguaggi imperativi, ispirandosi al paradigma della programmazione funzionale. \\

\begin{lstlisting}[language=kotlin,caption={Funzioni ordine superiore}]
fun esempio(str: String, fn: (String) -> String): Unit {
  val prova = fn(str)
  println(prova)
}
\end{lstlisting}

Java ha incominciato a includere le lambda nel suo linguaggio solo dalla versione 8, costringendo a tutti gli utilizzatori di versioni precedenti alla 8 di affidarsi a delle librerie esterne che forzavano l'utilizzo delle lambda su Java. \\
Kotlin  supporta le lambda nativamente senza l'utilizzo di librerie esterne e limitazioni aggiuntive.
Le lambda function e altri costrutti di linguaggi funzionali possono essere utili anche durante la programmazione di un applicazione Android, ove \'e richiesto ad esempio il passaggio di funzioni come parametro ad altre funzioni, chiamate asincrone verso un server esterno o localmente per interagire con l'interfaccia utente.

\begin{lstlisting}[language=kotlin,caption={Esempio Kotlin Programmazione funzionale}]
val lista: List = listOf("OS Windows", "Smartphone", "OS LINUX", "OS Android", "RAM", "OS IOS", "Scarpe")
println(strings.filter { it.startsWith("OS") }.map { it.toLowerCase() }.joinToString())
// "os windows, os linux, os android, os ios"

//lambda
val sum = { x: Int, y: Int -> x + y }
\end{lstlisting}



\section{Classi}
Kotlin come Java \'e un linguaggio orientato agli oggetti, oltre a mantenere i concetti fondamentali della programmazione ad oggetti, rimuove alcune verbosit\'a caratteristiche di molti linguaggi orientati agli oggetti.\\
Istanziare una classe in Kotlin \'e molto semplice e intuitivo poich\'e occorre chiamare direttamente il costruttore, senza dover utilizzare keywords aggiuntive come ``new'', presente in Java.

\begin{lstlisting}[language=java,caption={Esempio classe in Kotlin}]
val palla = Pallone(type="calcio", color="red")
\end{lstlisting}



\subsection{Data Class}

Nella programmazione Java e di altri linguaggi orientati agli oggetti vengono create classi per rappresentare modelli che verranno usati dall'applicazione per memorizzare valori o altri oggetti, questi modelli devono contenere i metodi get e set per leggere e settare i valori di un oggetto, Kotlin per rendere meno verbosa la creazione di classi, introduce il marcatore ``data'' permettendo al programmatore di scrivere solamente il costruttore senza dover pensare alla creazione dei metodi get, set, equals, toString, hashCode, tipicamente utilizzati nelle stesura di classi Java.\\
Questo rende il codice delle classi in Kotlin molto pi\'u coinciso e leggibile.

\begin{lstlisting}[language=java,caption={Esempio Data Class in Kotlin}]
data class User(val name: String, var password: String)
\end{lstlisting}

\begin{lstlisting}[language=java,caption={Esempio classe in Java}]
public class User {
 private String name;
 private String password;
 public User(String name, String password) {
  this.name = name;
  this.password = password;
 }
 public String getName() {
  return this.name;
 }
 public String getPassword() {
  return this.password;
 }
 public void setName(String name) {
  this.name = name;
 }
 public void setPassword(String password) {
  this.password = password;
 }
\end{lstlisting}




\section{Strutture e flussi di controllo}
Kotlin utilizza le pi\'u comuni strutture di controllo come if..else, try..catch, for e while introducendo il controllo "when" e aggiungendo funzionalit\'a aggiuntive rispetto a Java.

\begin{lstlisting}[language=kotlin,caption={Sintatti When Kotlin}]
when (x) {
    in 1..10 -> print("x is in the range")
    in validNumbers -> print("x is valid")
    !in 10..20 -> print("x is outside the range")
    x.isOdd() -> print("x is odd")
    else -> print("none of the above")
}
\end{lstlisting}
Un espressione \'e una dichiarazione che valuta una valore e ne restituisce un risultato, e si differeisce da una semplice dichiarazione che non restituisce nulla.

\begin{lstlisting}[language=kotlin,caption={Espressione - Flussi di Controllo}]
"hello".startsWith("h")
\end{lstlisting}

\begin{lstlisting}[language=kotlin,caption={Dischiarazione - Flussi di Controllo}]
val number = 2
\end{lstlisting}

Le comuni strutture di controllo in Java sono considerate dichiarazioni che non valutano e restituiscono nessun valore, in Kotlin invece, tutti i flussi di controllo restituiscono un valore.

\begin{lstlisting}[language=kotlin,caption={Esempio espresioni in Kotlin}]
//Flusso di Controllo Tradizionale
var max: Int
if (a > b)
  max = a
else
  max = b
// Flusso di controllo come Espressione
val max = if (a > b) a else b
\end{lstlisting}


\section{ Kotlin Android Extensions }
La struttura di un classico progetto Android comporta il continuo utilizzo di View Binding attraverso l'utilizzo di keyword come ``findViewById()'' che prende come parametro il riferimento alla View con la quale si vuole interagire. \\
Il continuo utilizzo questa funzione oltre a rendere poco leggibile il codice, provoca spesso numerosi bug dovuti ad un cattivo assegnamento di identificativi, Kotlin lasciando inalterato il funzionamento interno di findViewById, ha scelto di utilizzare un approccio simile a molte librerie\footnote{https://github.com/JakeWharton/butterknife} che cercarono di risolvere il problema della chiamata a findViewById ogni qualvolta si volesse interagire con un elemento di un layout.\\
Il funzionamento di queste librerie era molto semplice e intuitivo, utilizzavano le annotazioni offerte da Java per evitare codice inutile, bastava infatti fare riferimento all'ID della view e successivamente utilizzarla, in base al nome dato dal programmatore, infine in fase di compilazione veniva generato il codice con il findViewById().\\
La soluzione offerta da Kotlin risulta essere ancor pi\'u semplice ed intuitiva ed \'e: Kotlin Android Extensions, nata come una libreria assestante per poi essere integrata all'interno del linguaggio dalla versione 1.0, permettendo di fare riferimento ad una View scrivendo direttamente il suo ID, ed importando il riferimento al layout.

% \begin{lstlisting}[language=java,caption={Esempio Java}]
% TextView demo = findViewById(R.id.textView) as TextView;
% demo.setText("hello")
% \end{lstlisting}
%
% \begin{lstlisting}[language=java,caption={Esempio Java + Libreria esterna}]
% @BindView(R2.id.user) EditText username;
% username.setText("hello")
% \end{lstlisting}

\begin{lstlisting}[language=java,caption={Esempio KAE}]
import kotlinx.android.synthetic.main.activity_main.*
username.setText("hello");
\end{lstlisting}

\clearpage{\pagestyle{empty}\cleardoublepage}

\chapter{Firebase}                %crea il capitolo
%%%%%%%%%%%%%%%%%%%%%%%%%%%%%%%%%%%%%%%%%imposta l'intestazione di pagina
\lhead[\fancyplain{}{\bfseries\thepage}]{\fancyplain{}{\bfseries\rightmark}}
\section{Storia}                 %crea la sezione


Firebase è una piattaforma Mobile backend as a service (MBaaS) che consente
di interfacciare applicazioni mobili e web app ad un cloud backend, fornendo allo sviluppatore servizi utili per la gestione degli utenti, storage, notifiche push ed altri strumenti di analisi e sviluppo.\\
Il modello su cui si basa la piattaforma è relativamente recente poichè appoggiandosi al cloud computing, fornisce uno servizio globale e uniforme per connettere client differenti offrendo una sincronizzazione dei dati in tempo reale.\\
Lo sviluppo di Firebase iniziò dall'omonima azienda che nel 2011 sviluppò la piattaforma con l'idea di fornire un servizio in grado di sincronizzare dati in tempo reale, successivamente ricevette grande interessa da parte di Google che nel 2014 acquistò\footnote{https://techcrunch.com/2014/10/21/google-acquires-firebase-to-help-developers-build-better-realtime-apps/} Firebase e altre startup simili, integrandole con i suoi servizi Google Cloud Platform.


%https://gigaom.com/2013/06/20/firebase-gets-5-6m-to-launch-its-paid-product-and-fire-up-its-base/
% \begin{figure}[!hb]
%   \centering
%   \includegraphics[width=0.25\textwidth]{immagini/firebase.png}
%   \caption{Firebase logo.}
%   \label{fig:firebase logo}
% \end{figure}




\section{Servizi}                 %crea la sezione
Firebase offre diversi servizi, mettendo a disposizione anche SDK (Software Development Kit) e API (Application Programming Interface) multipiattaforma (Android, Ios, JavaScript, C++, Unity) per interagire con essi.\\
I servizi\footnote{https://firebase.google.com/products/} offerti da Firebase sono circa 20, realizzati per facilitare lo sviluppo e la gestione del backend, permettendo ad uno sviluppatore di concentrarsi maggiormente sulla parte client e meno sulla manutenzione e gestione del backend.\\
Tra i vari servizi forniti, quelli più utili, nell'ambito dello sviluppo software sono:


\begin{itemize}                         %crea un elenco puntato
\item \textbf{Firebase Cloud Messaging}: Soluzione cross-platform per la gestione di notifiche push su Android iOS e Web.

\item \textbf{Firebase Auth}: Servizio per la gestione degli utenti con il supporto del social login per Facebook, Github, Twitter, Google.

\item \textbf{Realtime Database}: Database NoSQL con il supporto della sincronizzazione in tempo reale dei dati fra diversi client.

\item \textbf{Firebase Storage}: Servizio che offre il trasferimento e l'hosting sicuro dei file.

\item \textbf{Firebase Hosting}: Web hosting che fornisce file utilizzando CDN (Content Delivery Network) e HTTP Secure (HTTPS).

\item \textbf{Cloud Functions}: Servizio che permette di eseguire script JavaScript ogni volta che vi è un cambiamento nel database.

\item \textbf{Firestore Database}: Database NoSQL basato su documenti con il supporto della sincronizzazione in tempo reale.
\end{itemize}




\section{Database}                 %crea la sezione
Google offre due differenti database con supporto della sincronizzazione dei dati in tempo reale:

\begin{itemize}
  \item \textbf{RealTime Database}
  \item \textbf{Cloud Firestore}
\end{itemize}


\textbf{RealTime Database} è un cloud database NoSQL che memorizza i dati in un unico file JSON. Il database è organizzato attraverso un modello gerarchico ad albero con un unica radice ed ha una struttura senza schema che può cambiare nel tempo.\\
Utilizzando l'SDK, tutte le richieste effettuate al RealTime Database vengono memorizzate localmente nella cache del client e vengono aggiornate in tempo reale al susseguirsi di modifiche ed eventi all'interno del Database.\\
Quando il dispositivo precedentemente offline riacquista la connessione, l'SDK del Realtime Database sincronizza le modifiche dei dati locali con gli aggiornamenti remoti che si sono verificati mentre il client era offline, risolvendo automaticamente eventuali conflitti.\\


\textbf{Cloud Firestore} è un cloud database NoSQL basato sulla memorizzazione dei dati sotto forma di documenti e collezioni, anche la sua struttura è senza schema e può quindi cambiare nel tempo.\\
Il modello di memorizzazione dei dati è basato sui documenti che possono contenere  stringhe e numeri, date, oggetti complessi e annidati.\\
Questi documenti sono archiviati in raccolte, chiamate collezioni che contengono i documenti, ma è anche possibile creare sub-collezioni all'interno dei documenti e creare strutture gerarchiche di dati che scalano man mano che il database cresce.\\
Gli unici limiti imposti da Firestore sono: la dimensione di un singolo documento che è di 1 MiB (1,048,576 bytes) e un massimo di 100 collezioni annidate.\\
L'SDK del database Firestore mantiene i dati aggiornati attraverso una buona gestione del caching, i client invece utilizzando appositi listener offrono una sincronizzazione dei dati in tempo reale.\\
L'aggiunta di listener oltre a informare il client su modifiche effettuate nel database, permette di memorizzare le richieste effettuate in precedenza e mantenere una copia delle risposte del server nella cache, offrendo quindi un supporto offline dei dati.

I tipi di dato messi a disposizione da Firestore sono:
\begin{table}[h]
\begin{center}
\begin{tabular}{|p{3cm}|p{10cm}|}
    \hline
\textbf{Tipo} & \textbf{Descrizione} \\ \hline
Array & non può contenere un altro valore array. \\ \hline
Boolean & falso, vero  \\ \hline
Date & Memorizzato in formato timestamp \\ \hline
Float & precisione numerica a 64 bit \\ \hline
Geo Point & Punto geografico contentente latitudine e longitudine \\ \hline
Integer & Intero Numerico a 64 bit \\ \hline
Map & Rappresenta un oggetto  \\ \hline
Null & valore nullo \\ \hline
Reference & Riferimento ad un'altro documento nel database  \\ \hline
String & Stringa di testo codificata in UTF-8.\\
\hline
\end{tabular}
\caption[Dati Firestore]{Tipi di dato Firestore}\label{tab:Firestore Tipi di dato}
\end{center}
\end{table}

L' interrogazione del database Firestore attraverso query risulta essere molto espressiva ed efficiente. La creazione delle query permette di filtrare i dati all'interno di un documento o filtrare collezioni, con la caratteristiche basilari delle interrogazioni: l'ordinamento, il filtraggio e limiti sui risultati di una query. Si possono filtrare anche sottocampi di un oggetto Map, ma non è possibile filtrare o ordinare un elemento di tipo ``Reference'' che serve solo ad indicare il riferimento di un documento all'interno del database.\\
Cloud Firestore offre un SDK con una buona integrazione per dispositivi mobili Android, iOS e web apps, ma permette l'utilizzo dei servizi anche offrendo SDK aggiuntive per altri linguaggi di programmazione come: NodeJS, Java, Python, e GO.\\
Ricapitolando, possiamo definire Firestore come una nuova versione di Firebase RealTime con una miglior struttura interna, una buona memorizzazione dei dati e una espressività delle query maggiore.

\begin{table}[h]
\begin{center}
\begin{tabular}{|p{7.5cm}|p{7cm}|}
    \hline
    \textbf{RealTime Firebase} & \textbf{Firestore} \\ \hline
    Memorizza i dati in un unico file JSON & Memorizza i dati in collezioni contententi documenti \\ \hline
    Supporto per i dati offline su Android e iOS & Supporto per i dati offline su Android, iOS e Web \\ \hline
    Depp Query con ordinamento e condizioni sui dati limitate & Query con ordinamento, condizioni sui dati, indicizzazione, alte performance \\ \hline
    Memorizzazione di dati come singole operazioni. & Memorizzazione e transizioni sui dati atomiche.\\ \hline
    Validazione dei dati manuale, e settaggio manuale di regole di protezione sui dati &  Validazione dei dati automatica, e regole di protezione sui dati manuali\\
\hline
\end{tabular}
\caption[Confronto tra Firebase e Firestore ]{Confronto dei due database Firebase}\label{tab:Confronto tra  Firestore e Firebase}
\end{center}
\end{table}


\subsection{Database Rules}                 %crea la sezione
Firebase offre per i suoi due database la possibilità di inserire delle restrizioni e regole di sicurezza per l'accesso al Database, chiamate Database Rules. Le Database Rules determinano chi ha accesso in lettura e scrittura al database o a collezioni di dati all'interno del database, queste regole sono gestite utilizzando il pannello di controllo di Firebase, una volta scritte le regole queste vengono applicate automaticamente ad ogni modifica. Ogni richiesta di lettura e scrittura di dati nel database sarà completata solo se le regole lo consentono.\\
Entrambi i database: Real time e Firestore supportano le Database Rules, le differenze fra i due servizi riguardano il metodo con cui vengono scritte le regole e il tipo di controlli che è possibile effettuare.\\
Firebase permette di scrivere le regole utilizzando un file in formato JSON dove vengono definite le regole in base alla collezione in cui si trovano i dati, alla validazione dei dati o in base all'utente registrato su Firebase Auth.\\
Le regole applicate al database RealTime hanno una sintassi simile a JavaScript e mettono a disposizione del programmatore quattro tipi di controlli:

\begin{table}[h]
\begin{center}
\begin{tabular}{|p{2cm}|p{12cm}|}
    \hline
    {\textbf{Tipo}} & {\textbf{Descrizione}} \\ \hline
    .read & Descrive se e quando i dati possono essere letti dagli utenti.\\ \hline
    .write & Descrive se e quando i dati possono essere scritti dagli utenti\\ \hline
     .validate & Definisce l'aspetto di un valore formattato correttamente, se ha attributi figli e il tipo di dato\\ \hline
    .indexOn & Specifica una collezione da indicizzare per supportare l'ordine e l' interrogazione\\ \hline
\end{tabular}
\caption[Firbase Rules ]{Firebase Database Rules}\label{tab:Firebase Database Rules}
\end{center}
\end{table}

Le regole per il database RealTime vengono memorizzate in formato JSON sui server Firebase, un esempio di alcune regole applicate è il seguente:

\begin{lstlisting}[language=javascript,caption={Firebase Rules esempio }]
{
  "rules": {
    "users": {
      "$uid": {
        ".write": "$uid === auth.uid"
      }
    },
    "collection2": {
     ".validate": "newData.isString() && newData.val().length < 100"
   }
  }
}
\end{lstlisting}


Le regole di sicurezza del database Firestore sono simili a quelle del RealTime Database ma prevedono un controllo degli accessi e della validazione dei dati in un formato più semplice ed espressivo.\\
Tutte le regole di sicurezza Firestore sono costituite da dichiarazioni di corrispondenza chiamate ``match'' che identificano i documenti nel database e consentono la creazione di espressioni che controllano l' accesso a tali documenti.\\
Oltre alle regole di scrittura, lettura, validazione è possibile creare funzioni ausiliarie per semplificare e rendere più intuitiva la scritture delle regole.\\
Un esempio di scrittura di alcune regole su un database Firestore è il seguente:

\begin{lstlisting}[language=javascript,caption={Firestore Database Rules}]
service cloud.firestore {
  match /databases/{database}/documents {
    function signedInOrPublic() {
      return request.auth.uid != null || resource.data.visibility == 'public';
   }
  match /cities/{city} {
      allow read: if request.auth.uid != null;
      allow create: if exists(/databases/$(database)/documents/users/$(request.auth.uid))
  }
 }
}
\end{lstlisting}



\section{Cloud Functions}                 %crea la sezione

Le Cloud Functions consentono di eseguire script backend in risposta a modifiche effettuate nel database Firestore o RealTime, o sullo storage Firebase. I linguaggi utilizzati per scrivere le Cloud Functions sono JavaScript e TypeScript, una volta scritta una funzione essa viene memorizzata e gestita dai server Firebase e man mano che il carico aumenta o diminuisce, Google scala automaticamente il numero di istanze di server virtuali necessari per eseguire le funzioni.


Il ciclo di vita di una funzione è il seguente:
\begin{itemize}
  \item Lo sviluppatore scrive il codice per una nuova funzione, selezionando un provider di eventi (Realtime Database, Firestore, Storage) e definisce le condizioni in cui la funzione deve essere eseguita.
  \item Lo sviluppatore tramite un tool a linea di comanda invia la funzione sui server di Firebase.
  \item Quando il provider dell'evento genera un evento che corrisponde alle condizioni della funzione, il codice della funzione viene eseguito.
  \item Se sono presenti più eventi da gestire contemporaneamente, Google creerà più instanze per gestire il lavoro più velocemente.
  \item Quando lo sviluppatore aggiorna la funzione distribuendo il codice aggiornato, tutte le istanze per la vecchia versione vengono cancellate e sostituite da nuove istanze.
  \item Quando uno sviluppatore cancella una funzione, tutte le istanze vengono cancellate e la connessione tra la funzione e il provider dell' evento viene rimossa.
\end{itemize}


           % ends first column but not page
Cloud Functions mette a disposizione quattro tipi di controllo sul database:

\begin{table}[h!]
\begin{tabular}{|p{2cm}|p{12cm}|}
    \hline
    \textbf{Evento} & \textbf{Trigger} \\ \hline
    onCreate & Attivato quando si scrive un documento per la prima volta.\\ \hline
    onUpdate & Attivato quando esiste già un documento e se ne modifica un valore.\\ \hline
    onDelete & Attivato quando un documento con dati viene eliminato.\\ \hline
    onWrite & Attivato quando si attiva onCreate, onUpdate o onDelete.\\ \hline

\end{tabular}
\caption[Firestore Rules]{Controlli Cloud Functions}\label{tab:Controlli Cloud Functions}
\end{table}

La stesura di una funzione viene effettuata definendo funzioni JavaScript o TypeScript ed indicando il documento o la collezione sui cui si vuole fare riferimento, seguito dal tipo di evento:


\begin{lstlisting}[language=javascript,caption={Cloud functions esempio 1 }]
exports.myFunctionName = functions.firestore.document('users/koci').onWrite((event) => {
    ...
  });
exports.modifyUser = functions.firestore.document('users/{userID}').onWrite(event => {
    //  documento sottoforma di oggetto
    var document = event.data.data();
    // documento precedente alla modificarlo
    var oldDocument = event.data.previous.data();
    ...
});
\end{lstlisting}


\newpage              % ends first column but not page
\subsection{Esempi di utilizzo}
Le cloud functions possono essere utilizzate anche interagendo con gli altri servizi Firebase, un esempio di integrazione tipico potrebbe essere quello di creare l'anteprima di un immagine e salvarla su Firebase Storage:

\begin{figure}[!h]
  \centering
  \includegraphics[width=0.5\textwidth]{immagini/functions_ex1.png}
  \caption{Firebase Storage e Cloud Functions esempio 1}\label{fig:Firebase Storage e Cloud Functions esempio 1}
\end{figure}

In questo esempio, una funzione del servizio Cloud Function rimane in ascolto su cambiamenti nello Storage in atttesa che un'immagine venga caricata, Quando un utente aggiungerà un immagine nello storage, verrà richiamata la funzione che scaricherà l'immagine e ne creerà una versione miniaturizzata, in seguito scriverà il riferimento della miniatura sul database, in modo che un'applicazione client possa trovarla e utilizzarla.
Un'altro esempio è l'utilizzo delle cloud functions per effettuare controlli su un tipo di linguaggio inappropriato all'interno di una chat,forum o commento:
la funzione esamina il testo, rimuove il linguaggio volgare e restituisce il testo revisionato.


\begin{figure}[!h]
\centering
  \includegraphics[width=0.5\textwidth]{immagini/functions_ex2.png}
  \caption{Firebase Storage e Cloud Functions esempio 2}
  \label{fig:Firebase Storage e Cloud Functions esempio 2}
\end{figure}


\newpage
%https://developer.xamarin.com/guides/android/data-and-cloud-services/google-messaging/firebase-cloud-messaging/
\section{Cloud Messaging}                 %crea la sezione
Firebase Cloud Messaging (FCM) è una soluzione di messaggistica multipiattaforma che consente di inviare messaggi tra dispositivi con la possibiltà, di notificare a una o più applicazioni client che sono disponibili nuovi dati.\\
è possibile inviare messaggi tramite l'Admin SDK\footnote{https://firebase.google.com/docs/admin/setup} o le API HTTP e XMPP, rispettando la dimensione massima di un messaggio corrispondente a 4KB.\\
I messaggi possono essere di due tipologie: messaggi di notifica e messaggi di dati, la differenza fra i due tipi di messaggi è il loro contenuto: i messaggi di notifica contengono un insieme predefinito di chiavi visualizzabili dall'utente, mentre i messaggi di dati contengono solo un insieme di chiave-valore definito da chi invia il messaggio.\\
Entrambi i messaggi richiedono di definire il campo obbligatorio ``token'' che è il riferimento ad un dispositivo o di un gruppo di dispositivi, questo token è generato tramite l'SDK di Firebase e deve essere memorizzato su un server per poter essere riutilizzato, in caso contrario l'SDK consente di aggiornare il token del dispositivo.

\begin{figure}[!hb]
  \centering
  \includegraphics[width=1\textwidth]{immagini/fcm_token.png}
  \caption{Processo di generazione di un token per il servizio FCM.}
  \label{fig:Processo di generazione di un token per il servizio FCM}
\end{figure}

\newpage
La ricezione dei messaggi di Cloud Messaging è possibile solo se si utilizza l'apposito SDK e si ottiene un token, necessario per inviare un messaggio ad un dispositivo specifico.\\
All'avvio iniziale dell'applicazione, l'SDK FCM genera un token di registrazione per l'istanza dell'applicazione client.\\
I token e la ricezione dei messaggi possono essere controllati utilizzando le funzioni offerte dall'SDK.
La classe ``FirebaseInstanceIdService'' viene utilizzata per la gestione dei token, mentre la classe ``FirebaseMessagingService'' viene utilizzata per la gestione dei messaggi ricevuti.\\
Il token di registrazione può cambiare quando l'applicazione elimina il token, ripristina, reinstalla, disnistalla l'app, o quando vengono eliminate le cache e i dati dal dispositivo.


\subsection{Messaggi}
I messaggi di dati inviabili tramite l'SDK o richieste HTTPS possono essere di due tipi:
\begin{itemize}
    \item \textbf{Downstream Messaging}
    \item \textbf{Upstream Messaging}
\end{itemize}

I messaggi Downstream, permettono di inviare una notifica push dal Server verso il Client

\begin{figure}[!hb]
  \centering
  \includegraphics[width=0.55\textwidth]{immagini/fcm_down.png}
  \caption{Esempio downstream messaging - Firebase Cloud Messaging}
  \label{fig:Esempio downstream messaging - Firebase Cloud Messaging}
\end{figure}


\begin{itemize}
    \item Il server (per esempio Cloud Functions) invia il messaggio a FCM.
    \item Se il client non è disponibile, il server FCM memorizza il messaggio in una coda per la successiva trasmissione. I messaggi vengono conservati nella memoria FCM per un massimo di 4 settimane.
    \item Quando il dispositivo client è disponibile, FCM inoltra il messaggio all'applicazione.
    \item L'applicazione client riceve il messaggio da FCM, lo elabora e lo visualizza all'utente.
\end{itemize}

In alternativa oltre all'invio di un messaggio ad un singolo dispositivo FCM consente la creazione di gruppi di utenti, permettendo con un singolo token di inviare un messaggio a tutti i dispositivi appartenenti ad un gruppo.\\
Il funzionamento dei messaggi Upstream invece è analogo e permette di inviare messaggi da un dispositivo client ad un server (ad esempio Cloud Functions).\\
Sulla base del modello di pubblicazione/iscrizione invece, FCM consente anche di inviare un messaggio a più dispositivi che si sono registrati ad un particolare argomento.

\begin{figure}[!hb]
  \centering
  \includegraphics[width=0.8\textwidth]{immagini/fcm_topic.png}
  \caption{Esempio topic messaging - Firebase Cloud Messaging}\label{fig:Esempio topic messaging - Firebase Cloud Messaging}
\end{figure}

    \begin{itemize}
        \item L'applicazione client si iscrive ad un argomento inviando un messaggio di sottoscrizione al server FCM.
        \item Il server invia messaggi tematici a FCM per la distribuzione.
        \item FCM inoltra messaggi tematici ai client che si sono registrati a tale argomento.
    \end{itemize}

Per iscriversi a un argomento, l'applicazione client chiama la seguente funzione:

\begin{lstlisting}[language=java,caption={FCM topic}]
 FirebaseMessaging.getInstance().subscribeToTopic("news");
\end{lstlisting}

Per annullare l'iscrizione, l'applicazione client deve richiamare ``unsubscribeFromTopic()'' con il nome del topic dal quale disiscriversi.

\subsection{Parametri Cloud Messaging}
Oltre al destinatario è possibile definire anche la priorità di un messaggio, la durata, il suono, l'icona, il tempo di vita e altri parametri opzionali.\\
I principali parametri messi a disposizione da Firebase sono:

\begin{table}[!h]
\begin{center}
\begin{tabular}{|l|p{11cm}|}
\hline
\textbf{Paramentro} & \textbf{Descrizione} \\ \hline
Title &	Titolo della notifica \\   \hline
Body &	Testo della notifica \\   \hline
Sound  &	Suono da riprodurre quando il dispositivo riceve la notifica \\   \hline
Sottotitolo  &	Sottotitolo della notifica \\   \hline
Icon & Icona della notifica \\   \hline
Timetolive & Specifica per quanto tempo (in secondi) il messaggio deve essere conservato se il dispositivo è offline  \\   \hline
Clickaction & L'azione associata al click della notifica \\   \hline

\end{tabular}
\caption[Cloud Messaging paramentri]{Cloud Messaging parametri}\label{tab:Cloud Messaging parametri}
\end{center}
\end{table}

\newpage
\subsection{Priorità}
Esistono due opzioni per assegnare la priorità di consegna ai messaggi: normale ed alta priorità. Il recapito di messaggi normali e di alta priorità funziona in questo modo:
\begin{itemize}
\item  \textbf{Priorità normale}: I messaggi vengono inviati immediatamente quando l'app è in primo piano, quando invece il dispositivo è in modalità Doze o l'applicazione è in standby la consegna potrebbe essere ritardata per risparmiare la batteria, i messaggi in questo caso richiedono di pianificare un Job FJD (Firebase Job Dispache) o un JobIntentService per gestire la notifica quando il dispositivo sarà nuovamente online.
\item \textbf{Alta priorità}: Il server Cloud Messaging tenta di inviare immediatamente il messaggio ad alta priorità, consentendo al servizio, attraverso l'SDK di cambiare lo stato del dispositivo e di eseguire alcune elaborazioni limitate (compreso un accesso alla rete molto limitato).
\end{itemize}





\section{FirebaseUI}                 %crea la sezione
FirebaseUI è un insieme di librerie open-source\footnote{https://firebaseopensource.com/projects/firebase/firebaseui-android/} disponibile per diverse piattaforme, che consentono di semplificare lo sviluppo di un applicazione che utilizza Firebase.\\
La libreria offre una versione semplificata per gestione dell'autenticazione, fornendo metodi che si integrano con i più comuni social, migliorando la comunicazione tra le View dell'interfaccia utente e il database, e facilita inoltre il collegamento e le richieste con il servizio Firebase Storage.\\
FirebaseUI dispone di moduli separati per utilizzare le varia librerie dedicate ai servizi:
\begin{itemize}
  \item  FirebaseUI Auth
  \item  FirebaseUI Firestore
  \item  FirebaseUI Database
  \item  FirebaseUI Storage
\end{itemize}


FirebaseUI-Auth cerca di gestire tutte le possibili casistiche che si possono riscontrare durante il login e la registrazione di un nuovo utente.
La libreria FirebaseUI-Auth offre una integrazione con i social login più diffusi (Google, Facebook, Twitter) e una buona integrazione con Smart Lock per memorizzare e recuperare le credenziali, consentendo l' accesso automatico e il single-tap sign-in, gestendo anche casi d'uso più complessi come il recupero dell'account e il collegamento di account multipli che sono sensibili alla sicurezza e difficili da implementare correttamente utilizzando le API di base fornite da Firebase.\\
FirebaseUI-Firestore semplifica la comunicazione e l'interazione dei dati fra Cloud Firestore e l'interfaccia utente dell' applicazione, fornendo un adapter personalizzato (FirestoreRecyclerAdapter) che consente la manipolazione e la sincronizzazione automatica dei dati, senza dover scrivere codice ripetitivo per ogni adapter che si interfaccia con il database.
\clearpage{\pagestyle{empty}\cleardoublepage}


\chapter{Architettura}                %crea il capitolo
%%%%%%%%%%%%%%%%%%%%%%%%%%%%%%%%%%%%%%%%%imposta l'intestazione di pagina
\lhead[\fancyplain{}{\bfseries\thepage}]{\fancyplain{}{\bfseries\rightmark}}
\pagenumbering{arabic}                  %mette i numeri arabi


L'applicazione \'e stata realizzata per la piattaforma mobile Android utilizzando il linguaggio di programmazione Kotlin, e altre librerie open-source. La parte client \'e stata scritta utilizzando il pattern MVP e utilizzando la classica organizzazione dei file di Android, differenziando quindi Activity, Fragment, Adapter e Servizi.\\
La parte server invece \'e stata realizzata utilizzando come BaaS Firebase e i suoi servizi offerti per la gestone del database, autenticazione, notifiche e storage.\\
Ogni servizio offerto da Firebase interagisce in maniera diretta o indiretta con tutti gli altri servizi,, che facilit\'a la gestisce dell'autenticazione, del database e dei uno storage online.\\


\newpage






\section{Server}                 %crea la sezione
La getione del backend \'e stata realizzata utilizzando la piattaforma Firebase e i suoi servizi.
I servizi utilizzati per la gestione del backend sono:
\begin{enumerate}
\item Auth: servizio per gestire l'autenticazione degli utenti
\item Firestore: database real-time per la memorizzazione di tutti i dati utilizzati nell'applicazione
\item Storage: spazio di archiviazione utilizzato per salvare gli avatar degli utenti e l'immagine principale dei gruppi.
\item Cloud Functions: servizio utilizzato per monitorare i cambiamenti all'interno di Firestore
\item Cloud Messaging: servizio utilizzato per gestire ed inviare notifiche ai dispositivi
\end{enumerate}



\begin{figure}[!hb]
  \centering
  \includegraphics[width=0.8\textwidth]{immagini/server_arch.png}
  \caption{Server Architettura}\label{fig:Architettura Server}
\end{figure}

\subsection{Autenticazione}
I client connessi a Firebase con l'appostia SDK, hanno la possibilit\'a di registrarsi attraverso email, e social login (Google,Faceook,Twitter). Una volta effettuata la registrazione, FirebaseAuth assegner\'a un identificativo univoco al nuovo client, memorizzando nei suoi server le informazioni basilari, quali: ID, nome, data di creazione, ultimo accesso, email, e provider (Email, Google, Faceook, Twitter).
\begin{figure}[!h]
  \centering
  \includegraphics[width=1\textwidth]{immagini/firebase_auth_user.png}
  \caption{Firebase Auth User}\label{fig:Firebase User}
\end{figure}

La parte di autenticazione viene gestita nel file AuthActivity.kt, che controlla se un utente \'e registrato o richiede di registrarsi.
Se l'utente richiede di registrarsi, l'interfaccia e la logica di registrazione vengono controllate dalla libreria FirebaseUI, l'accesso invece viene gestito manualmente.\\
Ogni utente \'e univoco e non pu\'o creare account diversi utilizzando la stessa email, inoltre utilizzando la libreria FirebaseUI si hanno ha disposizione l'integrazione con SmartLock, e l'account linking.
L'account linking consiste nel collegare account che utilizzano la stessa email, se si effettua ad esempio l'accesso attraverso uno dei social supportati, e l'email di registrazione del social \'e gi\'a presente nei server di FirebaseAuth, verr\'a effettuato un collegamento degli account automatico (Account Linking), fra gli account che utilizzano la stessa email.
Quando un utente registrato, effettua l'accesso, viene controllato se \'e presente il record all'interno del database Firestore, in caso contrario viene fatta richiesta di aggiungere il nuovo utente al database Firestore, utilizzando come ID, l'identificativo fornito dal servizio Firebase-Auth.Una volta effettuato l'accesso per evitare ulteriori richieste al Database vengono vengono salvate le informazioni dell'utente e gli identificativi dei membri appartenenti al gruppo nelle "Shared Preferences" di Android.


\subsection{Storage}

Il servizio di storage offerto da Firebase \'e stato utilizzato per salvare le immagini del profilo degli utenti registrati, e le immagini dei gruppi.\\
Il servizio offre la possibilit\'a di inserire qualsiasi tipo di file sia attraverso l'SDK per i vari client sia attraverso il pannello di controllo di Firebase.\\
Ogni file presente sullo storage contiene il nome, la dimensione, il tipo e l'ultima modifica del file, oltre a queste informazioni sono presenti due link: il riferimento del file sullo storage e un URL pubblico che consente il download del file.\\
Il riferimento viene utilizzato dall'SDK per avere un identificativo del file all'interno dello storage e quindi permettere il download del file applicando qual'ora ce ne fosse il bisogno restrizioni sul download, il link pubblico invece permette di visualizzare e scaricare il file a tutti coloro che ne possiedono il link (per motivi di sicurezza questo link pu\'o essere rigenerato).\\
I file sono organizzati in modo da avere le immagini del profilo degli utenti all interno di una cartella separata dalle immagini dei gruppi.\@
Si \'e scelto inoltre di memorizzare i file assegnandoli un identificativo univoco in modo da avere un riferimento di appartenenza sia attraverso il pannello di controllo di Firebase, sia utilizzando l'SDK.\\
Come nome per le immagini profilo, \'e stato usato l'identificativo offerto da FirebaseAuth, come nome per le immagini del gruppo invece \'e stato utilizzato l'identificativo del gruppo assegnato da Firestore.


\subsection{Notifiche}
Le notifiche vengono gestite dal server Cloud Messaging, che si occupa della memorizzazione, invio e ricezioni delle notifiche fra client e server.\\
La ricezione dei messaggi di Cloud Messaging \'e possibile solo se si utilizza l'apposito SDK e si ottiene un token, necessario per inviare un messaggio ad un dispositivo specifico.\\
All'avvio iniziale dell' applicazione, l'SDK FCM genera un token di registrazione per l'istanza dell' applicazione client.\\
I token e la ricezione dei messaggi possono essere controllati utilizzando le classi offerte dall'SDK.
La classe "FirebaseInstanceIdService" viene utilizzata per la gestione dei token, mentre la classe "FirebaseMessagingService" viene utilizzata per la gestione dei messaggi ricevuti.\\
Il token di registrazione pu\'o cambiare quando l'applicazione elimina il token, ripristina,riinstalla, dinistalla l'app, o quando vengono eliminare le cache e i dati dal dispositivo.


\subsection{Database}

\subsection{Cloud Functions}

\section{Client}                 %crea la sezione



\section{Model View Presenter}                 %crea la sezione
MVP (Model View Presenter) \'e un pattern architetturale utilizzato per l'organizzazione strutturale di un progetto, in modo da trarne vantaggio in termini di prestazioni, leggibilit\'a e modularit\'a del codice.\\
La sua caratteristica principale \'e quella di separare il livello di presentazione dalla logica, in modo che tutto ci\'o che riguarda l'interazione dell'utente con l'interfaccia sia separato da come vengono rappresentati i dati.\\
Il pattern MVP deriva dal pattern MVC (Model View Controller), che ha 3 concetti base, che lo definiscono:

\begin{enumerate}
\item Model: Il modello dei dati da visualizzare
\item View: L'interfaccia utente che visualizza i dati
\item Controller: Controlla l'interazione tra Model e View
\end{enumerate}

La principale differenza tra i due pattern \'e che il Presenter del MVP gestisce la logica tra la View e il Model, e la sua implementazione permette di gestire l'interfaccia utente ma soprattutto rendere pi\'u comoda l'interazione tra interfaccia utente e i dati.\\


Come il pattern MVC, anche il pattern MVP permette di rendere le View indipendenti dalla gestione dei dati, dividendo la logica dell' applicazione in tre livelli distinti, livelli che possono essere testati separatamente.\\
La possibilit\'a di poter testare i livelli separatamente \'e una delle caratteristiche del MVP.\@


\begin{enumerate}
\item Model: Il modello \'e un' interfaccia che definisce i dati da visualizzare.
\item View: La View \'e un' interfaccia passiva che visualizza i dati (il modello) e instrada i comandi utente (eventi) al Presenter per agire su tali dati.
\item Presenter: Il Presenter agisce sul modello e sulla vista. Recupera i dati dai repository (il modello) e li formatta per la visualizzazione nella vista.
\end{enumerate}

\begin{figure}[!h]
  \centering
  \includegraphics[width=0.65\textwidth]{immagini/mvc-vs-mvp.jpg}
  \caption{MVC vs MVP.}\label{fig:MVC vs MVP}
\end{figure}

\newpage


\subsection{Model}
Il Model \'e un'interfaccia dedicata all'acceso dei dati di un'applicazione, si occupa quindi di fornire un'astrazione del modello dei dati presenti nel database.\\
Il Model oltre a contenere la struttura dei dati da visualizzare si occupa anche di fornire una buona astrazione dei dati presenti nel database, modificando, aggiungendo e separando alcuni dei dati, in modo da rendere l'accesso e la visualizzazione dei dati pi\'u semplice per gli altri due componenti del pattern (View, Presenter).\\
Un esempio potrebbe essere il seguente:
Il database contiene una tabella con due tipi di dato:

\begin{enumerate}
\item \textbf{Nome}: String
\item \textbf{DataDiNascita}: Date
\end{enumerate}

Quando il programma ricever\'a i dati dal database in un qualsiasi formato( Map, Json, Array..) il Model selezionerà i dati in base alla definizione data dal programmatore trasformando il risultato del database in un oggetto.
Questo oggetto oltre a conserare le due informazioni ricevute dal database (Nome, DataDiNascita) potr'\a contenere ance informazioni aggiuntive inserite dal Model per facilitare l'uso e la manipolazione degli altri due componeti.
In questo caso il modello potrebbe creare il nuovo campo "et\'a" facendo una semplice sottrazione fra due date, quella attuale e la data di nascita dell'utente.

\begin{enumerate}
\item \textbf{Nome}: String
\item \textbf{DataDiNascita}: Date
\item \textbf{Et\'a}: Int
\end{enumerate}

%https://medium.com/@cervonefrancesco/model-view-presenter-android-guidelines-94970b430ddf

\subsection{View}
La View \'e un' interfaccia che definisce cosa deve implementare il Presenter, affinch\'e possa interagire con l'interfaccia utente.\\
La View interagisce con il Presenter per visualizzare i dati e notifica al Presenter le azioni che compie l'utente nell'interfaccia.\\
La View pu\'o essere implementata da un Activity, un Fragment, o un widget Android, che contengono ProgressBar, TextView, RecyclerView o altri elementi che necessitano di essere aggiornarnati in base a qualche azione dell'utente o cambiamento nel server.\\
Gli aggiornamenti della View possono essere gestiti in due diversi modi:
\begin{itemize}
    \item Passive View
    \item Supervising Controller
\end{itemize}

Nella Passive View, il Presenter aggiorna la vista per applicare i cambiamenti del modello, in questa modalit\'a l'interazione con il Model è gestita esclusivamente dal Presenter, la vista quindi ha un comportamento "passivo" e non è a conoscenza dei cambiamenti nel Model.\\
Ad esempio, se si dispone di un modulo username/password e di un pulsante "Invia", non si scrive la logica di validazione all' interno della View ma all' interno del Presenter. La View infatti dovrebbe solo contenere il nome utente e la password e inviarli al Presenter.

Nel Supervising Controller, la vista interagisce direttamente con il Model per eseguire semplici operazioni di binding dei dati, senza l' intervento del Presenter. Il Presenter aggiorna il Model, e gestissce cambiamenti sulla View solo nei casi pi\'u complessi, ad esempio l'aggiurnameto di un colore in base alle modifiche effettuate su un dato del Model, poich\'e la modifica non prevede una corrispondenza diretta tra la View e il Model
Entrambe le modalit\'a facilitano il testing delle view in Android poich\'e le loro implementazioni riducono al minimo la quantità di logica implementata nella View.

\begin{figure}[!hb]
  \centering
  \includegraphics[width=0.7\textwidth]{immagini/mvp_view_types.png}
  \caption{MVP View types}\label{fig:Model View Types}
\end{figure}

\subsection{Presenter}
Il Presenter \'e il mediatore tra il Model e la View e si occupa di  recuperare i dati dal Model, formattarli e passarli alla View, ma a differenza del pattern MVC, decide anche cosa succede quando si interagisce con la View reagendo alle interazioni dell'utente.\\
Il Presenter per facilitare il testing deve cercare di non dipendere minimamente da Android, ma contenere solo metodi e dipendente Java, senza l'utilizzo del "Context" ad esempio, questo permetter\'a di scrivere i test per il Presenter senza l'utilizzo di un emulatore Android.\\
Come detto in precedenza  il Presenter deve dipendere dall' interfaccia View e non direttamente dall' Activity o Fragment, in questo modo si tengono separati il Presenter e l'Activity rispettando la D dei principi SOLID:"Dipendi dalle astensioni. Non dipendere dalle concrezioni"\\





%%%%%%%%%%%%%%%%%%%%%%%%%%%%%%%%%%%%%%%%%non numera l'ultima pagina sinistra
\clearpage{\pagestyle{empty}\cleardoublepage}

\chapter{Applicazione}                %crea il capitolo
%%%%%%%%%%%%%%%%%%%%%%%%%%%%%%%%%%%%%%%%%imposta l'intestazione di pagina
\lhead[\fancyplain{}{\bfseries\thepage}]{\fancyplain{}{\bfseries\rightmark}}
\pagenumbering{arabic}                  %mette i numeri arabi


\section{Funzionalità}                 %crea la sezione
L'applicazione aiuta la gestione di attività e problemi riscontrati durante una convivenza fra due o più persone, in ambito lavorativo o fra studenti fuori-sede.\\
Le funzionalità principali consentono ai membri di un gruppo di gestire una lista di faccende comuni da svolgere, gestire e dividere le spese, organizzare eventi periodici e/o ricorrenti e confrontarsi utilizzando la chat di messaggistica istantanea.
L'applicazione avendo funzionalità molto generiche lascia il completo utilizzo di esse all'utente finale, permettendogli di gestire le varie funzionalità come meglio crede. Un esempio potrebbe essere la gestione degli eventi: alcuni studenti fuori sede potrebbero creare eventi per organizzare i turni di pulizia all'interno della casa, assegnando eventi ricorrenti a coinquilini specifici, in ambito lavorativo invece, i membri del gruppo potrebbero utilizzare la funzione di gestione degli eventi per organizzarsi il lavoro o creare incontri aziendali.\\
L'utente dopo aver effettuato l'accesso potrà interagire con le funzionalità dell'applicazione selezionando l'icona della relativa funzionalità dal menù.


\subsection{Gestione gruppo}
Effettuata la registrazione o l'accesso all'applicazione, l'utente avrà la possibilità di entrare a far parte di un gruppo o crearne uno nuovo, con l'eccezione che ogni utente può fare parte di un solo gruppo.\\
L'utente che sceglie di entrare a far parte di un nuovo gruppo deve aver precedentemente ricevuto il codice invito da un membro appartenente ad un gruppo esistente. Una volta ricevuto il codice di invito, il nuovo utente dovrà inserire il codice e confermare di entrare a far parte del gruppo, se conferma verranno aggiornati i membri del gruppo e il gruppo di appartenenza dell'utente e gli altri membri del gruppo invece riceveranno una notifica.\\
L'utente che sceglie invece di creare un nuovo gruppo deve indicare il nome del gruppo e un immagine opzionale da associarci, in seguito alla creazione potrà invitare altri utenti ad iscriversi al suo gruppo tramite un codice invito.

\subsection{Accesso e Registrazione}
Al primo avvio dell'applicazione, verranno mostrare delle pagine scorrevoli che illustreranno le caratteristiche principali con il quale può interagire l'utente, successivamente dopo una breve introduzione verrà mostrata la pagina di login, che permette di effettuare la registrazione e l'accesso attraverso un solo pulsante, senza differenziare se un utente sia già registrato o meno.\\
Quando l'utente cliccherà sul pulsane ``accedi'', l'applicazione automaticamente controllerà se l'utente si era già precedentemente registrato o deve effettuare la registrazione.\\
Se l'utente richiede di registrarsi, l'interfaccia e la logica di registrazione vengono controllate dalla libreria FirebaseUI, l'accesso invece viene gestito manualmente.\\
Ogni utente è univoco e non può creare account diversi utilizzando la stessa email, inoltre utilizzando la libreria FirebaseUI si hanno a disposizione l'integrazione con SmartLock, e l'account linking.
L'account linking consiste nel collegare account che utilizzano la stessa email, se si effettua, ad esempio l'accesso attraverso uno dei social supportati, e l'email di registrazione del social è già presente nei server di Firebase-Auth, verrà effettuato un collegamento degli account automatico (Account Linking), fra gli account che utilizzano la stessa email.\\
Quando un utente registrato, effettua l'accesso, viene controllato se è presente il record all'interno del database Firestore, in caso contrario viene fatta richiesta di aggiungere il nuovo utente al database Firestore, utilizzando come ID, l'identificativo fornito dal servizio Firebase-Auth.Una volta effettuato l'accesso per evitare ulteriori richieste al Database vengono anche salvate le informazioni dell'utente e gli identificativi dei membri appartenenti al gruppo nelle "Shared Preferences" di Android.\\
L'accesso e la registrazione possono essere effettuati utilizzando i social più diffusi o attraverso la semplice registrazione via email e password.\\
I social disponibili sono:
\begin{itemize}
  \item \textbf{Google Plus}
  \item \textbf{Facebook}
  \item \textbf{Twitter}
\end{itemize}
Se l'utente sceglierà di registrarsi attraverso l'utilizzo di un email, gli verranno richiesti l'email di registrazione, un nominativo (Nome,Cognome) e una password, per effettuare il login invece verranno richiesti solo l'email e la password.\\
Alternativamente se l'utente seleziona il metodo di registrazione attraverso un social, comparirà a schermo una finestra che chiederà all'utente registrato al social di consentire l'utilizzo dell'email e del nome dell'utente da parte dell'applicazione, una volta ricevuta l'autorizzazione, le volte successive verrà effettuato un login automatico senza richiedere permessi aggiuntivi.\\
Un utente che ha dimenticato la propria password può richiederne una nuova inserendo l'email di registrazione, in seguito dopo pochi secondi riceverà via email un avviso per reimpostare la password e un link che permetterà di reimpostare la password.\\



\subsection{Todolist}
Selezionando dal menù dell'applicazione l'icona della ``Todolist'', l'utente visualizzerà l'interfaccia dedicata per interagire con le funzionalità della todolist, quali: visualizzare le faccende da svolgere, visualizzare le faccende già svolte, aggiungere, modificare o eliminare una faccenda.\\
L'interfaccia per visualizzare le faccende è composta da due sezioni, la sezione delle faccende da completare, in primo piano e le faccende già completate in un'apposita sezione.\\
Le faccende sono composte da un nome obbligatorio, una data di scadenza, una priorità ed i membri del gruppo a cui è rivolta la faccenda.\\
Ogni utente visualizza la faccenda comprensiva di nome e data, la priorità invece viene indicata con un bordo colorato in base all'importanza della faccenda, le altre informazioni possono essere viste, cliccando sulla faccenda.\\
Gli utenti possono vedere sia le faccende create dagli altri membri del gruppo sia le loro faccende, in base alle restrizioni di visiblità assegnate durante la creazione, l'unica limitazione imposta riguarda le funzionalità di modifica ed eliminazione, che sono consentite solamente all'utente che ha creato la faccenda, gli altri utenti invece potranno completare la faccenda marcandola.\\
Le faccende che vengono marcate e completate vengono spostate automaticamente nell'elenco delle faccende completate e ogni utente avrà la posibilità di portare nuovamente una faccenda non completata nella sezione delle faccende da completare, senza dover aggiungerne un'ulteriore. Quando si sposta una faccenda dalla sezione "faccende completate" alla sezione "faccende da completare" il nome, la visibilità e la priorità della faccenda rimarranno inalterate, mentre la data di scadenza sarà impostata al giorno in cui è avvenuto il camiamento.\\
L'aggiunta di una nuova faccenda viene effettuata attraverso due modalità differenti: la prima rapida, la seconda personalizzata.\\
La modalità rapida si trova nella parte superiore dello schermo, sottostante alla toolbar, in questa modalità l'utente può aggiungere un nuovo elemento indicando solamente il nome ed in automatico l'applicazione setterà i campi opzionali, impostando la data di scadenza alla data in cui è stato aggiunto l'elemento, la priorità di medio livello e la visibilità a tutti i membri del gruppo.\\
La modalità personalizzata invece permette di inserire tutte le informazioni possibili per una faccenda, questa modalità di aggiunta compare se l'utente clicca la relativa icona presente nella toolbar della pagina "Todolist". Una volta cliccata l'icona si aprirà una finestra con un testo da completare corrispondente al nome della faccenda e tre icone: l'icona di una data, l'icona di un gruppo e l'icona della priorità, che se cliccate consentono all'utente di inserire le informazioni opzionali.
\begin{itemize}
    \item \textbf{Priorità:} la priorità di una faccenda dispone di 3 opzioni:``Bassa priorità'', ``Alta priorità'' e ``Media priorità''.
    \item \textbf{Visibilità:} la visibilità di una faccenda si potrà indicare selezionando da una lista gli utenti del gruppo.
    \item \textbf{Data:} la data viene impostata, attraverso un calendario, indicando il giorno di scadenza della faccenda.
\end{itemize}




\subsection{Spese}
La seconda funzionalità principale dell'applicazione è la gestione delle spese condivise, per accedere a questa funzionalità l'utente dovrà cliccare l'icona di un portafoglio dal relativo menù delle funzionalità.\\
L'interfaccia che si presenta all'utente è molto simile all'interfaccia della gestione delle faccende: è presente la visualizzazione globale delle spese da pagare e pagate e la possibilità di aggiungerne modificarne o cancellarne una.\\
Ogni utente appartenente al gruppo visualizzerà tutte le spese non completate e quelle già completate e in qualsiasi momento potrà marcare una spesa, segnandola come "pagata".
La visualizzazione dei una singola spesa comprende di nome, la data di scadenza, l'icona della categoria a cui è associata la spesa, e la quota parziale che dovrà pagare l'utente. Cliccando su una spesa apparirà una finestra di dialogo che mostrerà il resoconto totale della spesa con la lista degli utenti che hanno pagato la loro quota e la lista degli utenti che ancora devono pagarla. Marcando una spesa l'utente segnerà di aver pagato la sua quota e di conseguenza la spesa, per quell'utente, verrà spostate automaticamente nell'elenco delle spese pagate.\\
Le funzionalità di modifica ed eliminazione di una spesa sono consentite solamente all'utente che ha creato la spesa, gli altri utenti invece potranno solamente indicare di aver pagato la quota, marcandola.\\
Le modalità di aggiunta di una nuova spesa sono due, la modalità rapida e la modalità personalizzata.\\
La modalità rapida è accessibile attraverso l'interfaccia principale, nella parte superiore dello schermo infatti sono presenti due caselle di testo e un pulsante.Questa modalità permette di indicare solamente i parametri obbligatori: il nome e l'ammontare globale, alternativamente se l'utente vuole specificare anche altre informazioni, dovrà utilizzare il relativo pulsante per accedere alla finestra con tutti i campi opzionali per la creazione di una spesa.\\
La modalità di aggiunta personalizzata invece prevede un interfaccia con una casella di testo corrispondente al nome della spesa, e un'altra casella di testo corrispondente all'ammontare globale, sottostante alle caselle ci saranno tre icone: l'icona di un calendario, l'icona di un file, l'icona di un gruppo e l'icona di un etichetta, che se cliccate consentiranno all'utente di inserire le informazioni opzionali.

\begin{itemize}
   \item \textbf{Descrizione:} casella di testo per indicare una descrizone della spesa
   \item \textbf{Visibilità:} scelta multipla fra gli utenti del gruppo per indicare a chi è rivolta la spesa.
   \item \textbf{Data:} calendario per indicanre il giorno di scadenza della spesa.
   \item \textbf{Categoria:} scelta multipla per indicare il la tipologia di spesa effettuata
\end{itemize}
Le categorie selezionabili sono preimpostate dall'applicazione e sono le seguenti:
\begin{itemize}
    \item Bolletta Gas
    \item Bolletta Acqua
    \item Bolletta Luce
    \item Bolletta Telefono
    \item Cibo
    \item Pulizie
    \item Casa
    \item Strumenti
    \item Spesa generica
\end{itemize}




\subsection{Chat}
L'applicazione offre una chat di messaggistica istantanea integrata che consente di comunicare con tutti i membri appartenenti al gruppo in tempo reale.\\
L'interfaccia della sezione chat è simile ad altre applicazioni di messaggistica istantanea e consente di visualizzare tutti i messaggi inviati dall'utente e ricevuti dagli altri membri del gruppo.\\
I messaggi inviati dall'utente saranno contrassegnati con un colore blu e si troveranno nella parte destra dello schermo, mentre i messaggi ricevuto dagli altri membri del gruppo si troveranno nella parte sinistra dello schermo con un color differente e informazioni aggiuntive come il nome e l'avatar dell'utente che ha inviato il messaggio.
Nella parte inferiore dello schermo è presente una casella di testo e un pulsante consentendo all'utente di scrivere e inviare un nuovo messaggio che sarà spedito in tempo reale a tutti i membri del gruppo, infatti una volta inviati, i messaggi appariranno come notifica a tutti i dispositivi online.\\
Quado un utente non dispone di una connessione internet, il messaggio verrà conservato e l'utente verrà notificato appena si connetterà ad internet.


\subsection{Eventi}
L'ultima funzionalità è la gestione degli eventi, che permetterà ai membri del gruppo di creare degli eventi indicando una data, e qualora l'evento fosse ricorrente indicando la ricorrenza dell'evento.\\
L'interfaccia della pagina dedicata agli eventi è molto semplice, sulla parte superiore della toolbar è presente un icona che permette di aggiungere un nuovo evento, sottostante ad essa sono presenti tutti gli eventi sottoforma di lista.\\
Quando un utente clicca sull'icona per inserire un nuovo evento, verrà mostrata una finestra di dialogo, dove l'utente dovrà indicare il nome dell'evento, un eventuale descrizione, i partecipanti all'evento, e la ricorrenza dell'evento. La ricorrenza dell'evento può essere di cinque tipi differenti:
\begin{itemize}
  \item Non ripetere
  \item Giornaliera
  \item Settimanale
  \item Mensile
  \item Annuale
\end{itemize}
Una volta selezionata la ricorrenza verrà richiesta la data di riferimento dell'evento.\\




\subsection{Menù}
L'applicazione offre due menù differenti, il menù delle funzionalità e il menù delle impostazioni.\\
Il menù delle funzionalità si trova nella parte inferiore dello schermo, mentre per accedere al menù delle impostazioni, bisognerà cliccare la relativa icona del menù presente nella toolbar.\\
Interagendo con il menù delle impostazioni laterale si potranno gestire le informazioni personali dell'utente e gestire le informazioni del suo gruppo.\\
Nella pagina riguardante il profilo dell'utente sarà possibile visualizzare le informazioni personali come il nome, l'email e il gruppo a cui esso appartiene, queste informazioni sono modificabili in qualsiasi momento.\\
Nella pagina riguardante il gruppo invece sarà possibile visualizzare le informazioni principali riguardanti il gruppo a cui l'utente appartiene, come: il nome, l'immagine e gli utente che appartengono al gruppo. Le informazioni modificabili in questa pagina sono l'immagine del gruppo, il nome del gruppo e la possibilità di invitare altre persone ad unirsi al gruppo tramite invito, (verrà inviato all'utente un codice di invito che dovrà inserire al momento della registrazione).\\

\include{Appendice/AppendiceA}

%%%%%%%%%%%%%%%%%%%%%%%%%%%%%%%%%%%%%%%%%per fare le conclusioni
\chapter*{Conclusioni}
%%%%%%%%%%%%%%%%%%%%%%%%%%%%%%%%%%%%%%%%%imposta l'intestazione di pagina
\rhead[\fancyplain{}{\bfseries
Conclusioni}]{\fancyplain{}{\bfseries\thepage}}
\lhead[\fancyplain{}{\bfseries\thepage}]{\fancyplain{}{\bfseries
CONCLUSIONI}}
%%%%%%%%%%%%%%%%%%%%%%%%%%%%%%%%%%%%%%%%%aggiunge la voce Conclusioni
                                        %   nell'indice
\addcontentsline{toc}{chapter}{Conclusioni}
Kotlin\footnote{https://kotlinlang.org/}.
Conclusione bla bla.\\
blabla blablablabla blabla

\clearpage{\pagestyle{empty}\cleardoublepage}

%

\begin{thebibliography}{9}

\bibitem{latexcompanion}
Massimo Carli
\textit{Android 6. Guida per lo sviluppatore}.
Apogeo, 2016.

\bibitem{latexcompanion}
Stephen Samuel, Stefan Bocutiu
\textit{Programming Kotlin}.
Packtpub, 2017.

\bibitem{latexcompanion}
 Miloš Vasić
 \textit{Mastering Android Development with Kotlin}.
Packtpub, 2017.


\bibitem{latexcompanion}
Ronak K. Panchal,Akshay K. Patel \textit{A comparative study: Java Vs kotlin Programming in Android}.
International Journal of Innovative Trends in Engineering and Research, 2016.

\bibitem{ReactiveX}
ReactiveX documentation
\\\texttt{http://reactivex.io/documentation}

\bibitem{Android Developer }
Google Android Developer Documentation
\\\texttt{https://developer.android.com}

\bibitem{Firebase}
Google Firebase Documentation
\\\texttt{https://firebase.google.com/docs/}

\bibitem{Google I/O}
Google I/O - 2017
\\\texttt{https://events.google.com/io/}

\bibitem{Kotlin Lang}
Kotlin Language Documentation
\\\texttt{https://kotlinlang.org/docs/kotlin-docs.pdf}

\bibitem{Java Docs}
Oracle Java Documentation
\\\texttt{https://docs.oracle.com/}

\bibitem{Xamarin Guides}
Xamarin Android Google services guides
\\\texttt{https://developer.xamarin.com/guides/android/data-and-cloud-services/}

\bibitem{FirebaseUI-Android}
Libreria per la gestione dei servizi Firebase
\\\texttt{https://github.com/firebase/FirebaseUI-Android}

\bibitem{RxJava}
Libreria di supporto per la programmazione reattiva su Java
\\\texttt{https://github.com/ReactiveX/RxJava}

\bibitem{RxKotlin}
Libreria di supporto per la programmazione reattiva su Kotlin
\\\texttt{https://github.com/ReactiveX/RxAndroid}

\bibitem{RxAndroid}
Libreria di supporto per la programmazione reattiva su Android
\\\texttt{https://github.com/ReactiveX/RxKotlin}

\bibitem{BottomNavigationViewEx}
Libreria di supporto per il Widget del menu di navigazione
\\\texttt{https://github.com/ittianyu/BottomNavigationViewEx}

\bibitem{Welcome Android}
Libreria di supporto per mostrare pagine di presentazione e introduzione iniziali
\\\texttt{https://github.com/stephentuso/welcome-android}

\bibitem{MoneyEditText}
Libreria di supporto per il widget che consente di indicare l'ammontare di una spesa
\\\texttt{https://github.com/shuhart/MoneyEditText}

\bibitem{LoadingButtonAndroid }
Libreria di supporto per il widget che trasforma un Button in una barra di caricamento
\\\texttt{https://github.com/leandroBorgesFerreira/LoadingButtonAndroid}

\bibitem{Glide}
Libreria di supporto per la gestione delle immagini con funzionalità di caching
\\\texttt{https://github.com/bumptech/glide}

\bibitem{CircleImageView}
Libreria di supporto per visualizzare immagini con un bordo circolare
\\\texttt{https://github.com/hdodenhof/CircleImageView}

\bibitem{SublimePicker}
Libreria di supporto per la selezione di una data da un calendario personalizzabile
\\\texttt{https://github.com/vikramkakkar/SublimePicker/}

\bibitem{ImagePicker}
Libreria di supporto per selezionare un'immagine dalla galleria del dispositivo
\\\texttt{https://github.com/Mariovc/ImagePicker}

\bibitem{facebook Android SDK}
Libreria di supporto per l'accesso tramite il social network Facebook
\\\texttt{https://github.com/facebook/facebook-android-sdk}


\bibitem{Firebase Command Line Tools}
Tool a linea di comando per gestire i servizi Firebase
\\\texttt{https://github.com/facebook/facebook-android-sdk}


\bibitem{Firebase SDK for Cloud Functions}
Libreria di supporto per l'utilizzo del servizio Firebase Cloud Functions
\\\texttt{https://github.com/firebase/firebase-functions}




\end{thebibliography}



\chapter*{Ringraziamenti}
\thispagestyle{empty}
Un ringraziamento speciale alla mia famiglia, parenti e amici che mi ha sempre sostenuto in questo percorso.


\end{document}
