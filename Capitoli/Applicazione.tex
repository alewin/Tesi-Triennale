\chapter{Applicazione}                %crea il capitolo
%%%%%%%%%%%%%%%%%%%%%%%%%%%%%%%%%%%%%%%%%imposta l'intestazione di pagina
\lhead[\fancyplain{}{\bfseries\thepage}]{\fancyplain{}{\bfseries\rightmark}}
\pagenumbering{arabic}                  %mette i numeri arabi


\section{Funzionalità}                 %crea la sezione
L'applicazione aiuta la gestione di attività e problemi riscontrati durante una convivenza fra due o più persone, in ambito lavorativo o fra studenti fuori-sede.\\
Le funzionalità principali consentono ai membri di un gruppo di gestire una lista di faccende comuni da svolgere, gestire e dividere le spese, organizzare eventi periodici e/o ricorrenti e confrontarsi utilizzando la chat di messaggistica istantanea.
L'applicazione avendo funzionalità molto generiche lascia il completo utilizzo di esse all'utente finale, permettendogli di gestire le varie funzionalità come meglio crede. Un esempio potrebbe essere la gestione degli eventi: alcuni studenti fuori sede potrebbero creare eventi per organizzare i turni di pulizia all'interno della casa, assegnando eventi ricorrenti a coinquilini specifici, in ambito lavorativo invece, i membri del gruppo potrebbero utilizzare la funzione di gestione degli eventi per organizzarsi il lavoro o creare incontri aziendali.\\
L'utente dopo aver effettuato l'accesso potrà interagire con le funzionalità dell'applicazione selezionando l'icona della relativa funzionalità dal menù.


\subsection{Gestione gruppo}
Effettuata la registrazione o l'accesso all'applicazione, l'utente avrà la possibilità di entrare a far parte di un gruppo o crearne uno nuovo, con l'eccezione che ogni utente può fare parte di un solo gruppo.\\
L'utente che sceglie di entrare a far parte di un nuovo gruppo deve aver precedentemente ricevuto il codice invito da un membro appartenente ad un gruppo esistente. Una volta ricevuto il codice di invito, il nuovo utente dovrà inserire il codice e confermare di entrare a far parte del gruppo, se conferma verranno aggiornati i membri del gruppo e il gruppo di appartenenza dell'utente e gli altri membri del gruppo invece riceveranno una notifica.\\
L'utente che sceglie invece di creare un nuovo gruppo deve indicare il nome del gruppo e un immagine opzionale da associarci, in seguito alla creazione potrà invitare altri utenti ad iscriversi al suo gruppo tramite un codice invito.

\subsection{Accesso e Registrazione}
Al primo avvio dell'applicazione, verranno mostrare delle pagine scorrevoli che illustreranno le caratteristiche principali con il quale può interagire l'utente, successivamente dopo una breve introduzione verrà mostrata la pagina di login, che permette di effettuare la registrazione e l'accesso attraverso un solo pulsante, senza differenziare se un utente sia già registrato o meno.\\
Quando l'utente cliccherà sul pulsane ``accedi'', l'applicazione automaticamente controllerà se l'utente si era già precedentemente registrato o deve effettuare la registrazione.\\
Se l'utente richiede di registrarsi, l'interfaccia e la logica di registrazione vengono controllate dalla libreria FirebaseUI, l'accesso invece viene gestito manualmente.\\
Ogni utente è univoco e non può creare account diversi utilizzando la stessa email, inoltre utilizzando la libreria FirebaseUI si hanno a disposizione l'integrazione con SmartLock, e l'account linking.
L'account linking consiste nel collegare account che utilizzano la stessa email, se si effettua, ad esempio l'accesso attraverso uno dei social supportati, e l'email di registrazione del social è già presente nei server di Firebase-Auth, verrà effettuato un collegamento degli account automatico (Account Linking), fra gli account che utilizzano la stessa email.\\
Quando un utente registrato, effettua l'accesso, viene controllato se è presente il record all'interno del database Firestore, in caso contrario viene fatta richiesta di aggiungere il nuovo utente al database Firestore, utilizzando come ID, l'identificativo fornito dal servizio Firebase-Auth.Una volta effettuato l'accesso per evitare ulteriori richieste al Database vengono anche salvate le informazioni dell'utente e gli identificativi dei membri appartenenti al gruppo nelle "Shared Preferences" di Android.\\
L'accesso e la registrazione possono essere effettuati utilizzando i social più diffusi o attraverso la semplice registrazione via email e password.\\
I social disponibili sono:
\begin{itemize}
  \item \textbf{Google Plus}
  \item \textbf{Facebook}
  \item \textbf{Twitter}
\end{itemize}
Se l'utente sceglierà di registrarsi attraverso l'utilizzo di un email, gli verranno richiesti l'email di registrazione, un nominativo (Nome,Cognome) e una password, per effettuare il login invece verranno richiesti solo l'email e la password.\\
Alternativamente se l'utente seleziona il metodo di registrazione attraverso un social, comparirà a schermo una finestra che chiederà all'utente registrato al social di consentire l'utilizzo dell'email e del nome dell'utente da parte dell'applicazione, una volta ricevuta l'autorizzazione, le volte successive verrà effettuato un login automatico senza richiedere permessi aggiuntivi.\\
Un utente che ha dimenticato la propria password può richiederne una nuova inserendo l'email di registrazione, in seguito dopo pochi secondi riceverà via email un avviso per reimpostare la password e un link che permetterà di reimpostare la password.\\



\subsection{Todolist}
Selezionando dal menù dell'applicazione l'icona della ``Todolist'', l'utente visualizzerà l'interfaccia dedicata per interagire con le funzionalità della todolist, quali: visualizzare le faccende da svolgere, visualizzare le faccende già svolte, aggiungere, modificare o eliminare una faccenda.\\
L'interfaccia per visualizzare le faccende è composta da due sezioni, la sezione delle faccende da completare, in primo piano e le faccende già completate in un'apposita sezione.\\
Le faccende sono composte da un nome obbligatorio, una data di scadenza, una priorità ed i membri del gruppo a cui è rivolta la faccenda.\\
Ogni utente visualizza la faccenda comprensiva di nome e data, la priorità invece viene indicata con un bordo colorato in base all'importanza della faccenda, le altre informazioni possono essere viste, cliccando sulla faccenda.\\
Gli utenti possono vedere sia le faccende create dagli altri membri del gruppo sia le loro faccende, in base alle restrizioni di visiblità assegnate durante la creazione, l'unica limitazione imposta riguarda le funzionalità di modifica ed eliminazione, che sono consentite solamente all'utente che ha creato la faccenda, gli altri utenti invece potranno completare la faccenda marcandola.\\
Le faccende che vengono marcate e completate vengono spostate automaticamente nell'elenco delle faccende completate e ogni utente avrà la posibilità di portare nuovamente una faccenda non completata nella sezione delle faccende da completare, senza dover aggiungerne un'ulteriore. Quando si sposta una faccenda dalla sezione "faccende completate" alla sezione "faccende da completare" il nome, la visibilità e la priorità della faccenda rimarranno inalterate, mentre la data di scadenza sarà impostata al giorno in cui è avvenuto il camiamento.\\
L'aggiunta di una nuova faccenda viene effettuata attraverso due modalità differenti: la prima rapida, la seconda personalizzata.\\
La modalità rapida si trova nella parte superiore dello schermo, sottostante alla toolbar, in questa modalità l'utente può aggiungere un nuovo elemento indicando solamente il nome ed in automatico l'applicazione setterà i campi opzionali, impostando la data di scadenza alla data in cui è stato aggiunto l'elemento, la priorità di medio livello e la visibilità a tutti i membri del gruppo.\\
La modalità personalizzata invece permette di inserire tutte le informazioni possibili per una faccenda, questa modalità di aggiunta compare se l'utente clicca la relativa icona presente nella toolbar della pagina "Todolist". Una volta cliccata l'icona si aprirà una finestra con un testo da completare corrispondente al nome della faccenda e tre icone: l'icona di una data, l'icona di un gruppo e l'icona della priorità, che se cliccate consentono all'utente di inserire le informazioni opzionali.
\begin{itemize}
    \item \textbf{Priorità:} la priorità di una faccenda dispone di 3 opzioni:``Bassa priorità'', ``Alta priorità'' e ``Media priorità''.
    \item \textbf{Visibilità:} la visibilità di una faccenda si potrà indicare selezionando da una lista gli utenti del gruppo.
    \item \textbf{Data:} la data viene impostata, attraverso un calendario, indicando il giorno di scadenza della faccenda.
\end{itemize}




\subsection{Spese}
La seconda funzionalità principale dell'applicazione è la gestione delle spese condivise, per accedere a questa funzionalità l'utente dovrà cliccare l'icona di un portafoglio dal relativo menù delle funzionalità.\\
L'interfaccia che si presenta all'utente è molto simile all'interfaccia della gestione delle faccende: è presente la visualizzazione globale delle spese da pagare e pagate e la possibilità di aggiungerne modificarne o cancellarne una.\\
Ogni utente appartenente al gruppo visualizzerà tutte le spese non completate e quelle già completate e in qualsiasi momento potrà marcare una spesa, segnandola come "pagata".
La visualizzazione dei una singola spesa comprende di nome, la data di scadenza, l'icona della categoria a cui è associata la spesa, e la quota parziale che dovrà pagare l'utente. Cliccando su una spesa apparirà una finestra di dialogo che mostrerà il resoconto totale della spesa con la lista degli utenti che hanno pagato la loro quota e la lista degli utenti che ancora devono pagarla. Marcando una spesa l'utente segnerà di aver pagato la sua quota e di conseguenza la spesa, per quell'utente, verrà spostate automaticamente nell'elenco delle spese pagate.\\
Le funzionalità di modifica ed eliminazione di una spesa sono consentite solamente all'utente che ha creato la spesa, gli altri utenti invece potranno solamente indicare di aver pagato la quota, marcandola.\\
Le modalità di aggiunta di una nuova spesa sono due, la modalità rapida e la modalità personalizzata.\\
La modalità rapida è accessibile attraverso l'interfaccia principale, nella parte superiore dello schermo infatti sono presenti due caselle di testo e un pulsante.Questa modalità permette di indicare solamente i parametri obbligatori: il nome e l'ammontare globale, alternativamente se l'utente vuole specificare anche altre informazioni, dovrà utilizzare il relativo pulsante per accedere alla finestra con tutti i campi opzionali per la creazione di una spesa.\\
La modalità di aggiunta personalizzata invece prevede un interfaccia con una casella di testo corrispondente al nome della spesa, e un'altra casella di testo corrispondente all'ammontare globale, sottostante alle caselle ci saranno tre icone: l'icona di un calendario, l'icona di un file, l'icona di un gruppo e l'icona di un etichetta, che se cliccate consentiranno all'utente di inserire le informazioni opzionali.

\begin{itemize}
   \item \textbf{Descrizione:} casella di testo per indicare una descrizone della spesa
   \item \textbf{Visibilità:} scelta multipla fra gli utenti del gruppo per indicare a chi è rivolta la spesa.
   \item \textbf{Data:} calendario per indicanre il giorno di scadenza della spesa.
   \item \textbf{Categoria:} scelta multipla per indicare il la tipologia di spesa effettuata
\end{itemize}
Le categorie selezionabili sono preimpostate dall'applicazione e sono le seguenti:
\begin{itemize}
    \item Bolletta Gas
    \item Bolletta Acqua
    \item Bolletta Luce
    \item Bolletta Telefono
    \item Cibo
    \item Pulizie
    \item Casa
    \item Strumenti
    \item Spesa generica
\end{itemize}




\subsection{Chat}
L'applicazione offre una chat di messaggistica istantanea integrata che consente di comunicare con tutti i membri appartenenti al gruppo in tempo reale.\\
L'interfaccia della sezione chat è simile ad altre applicazioni di messaggistica istantanea e consente di visualizzare tutti i messaggi inviati dall'utente e ricevuti dagli altri membri del gruppo.\\
I messaggi inviati dall'utente saranno contrassegnati con un colore blu e si troveranno nella parte destra dello schermo, mentre i messaggi ricevuto dagli altri membri del gruppo si troveranno nella parte sinistra dello schermo con un color differente e informazioni aggiuntive come il nome e l'avatar dell'utente che ha inviato il messaggio.
Nella parte inferiore dello schermo è presente una casella di testo e un pulsante consentendo all'utente di scrivere e inviare un nuovo messaggio che sarà spedito in tempo reale a tutti i membri del gruppo, infatti una volta inviati, i messaggi appariranno come notifica a tutti i dispositivi online.\\
Quado un utente non dispone di una connessione internet, il messaggio verrà conservato e l'utente verrà notificato appena si connetterà ad internet.


\subsection{Eventi}
L'ultima funzionalità è la gestione degli eventi, che permetterà ai membri del gruppo di creare degli eventi indicando una data, e qualora l'evento fosse ricorrente indicando la ricorrenza dell'evento.\\
L'interfaccia della pagina dedicata agli eventi è molto semplice, sulla parte superiore della toolbar è presente un icona che permette di aggiungere un nuovo evento, sottostante ad essa sono presenti tutti gli eventi sottoforma di lista.\\
Quando un utente clicca sull'icona per inserire un nuovo evento, verrà mostrata una finestra di dialogo, dove l'utente dovrà indicare il nome dell'evento, un eventuale descrizione, i partecipanti all'evento, e la ricorrenza dell'evento. La ricorrenza dell'evento può essere di cinque tipi differenti:
\begin{itemize}
  \item Non ripetere
  \item Giornaliera
  \item Settimanale
  \item Mensile
  \item Annuale
\end{itemize}
Una volta selezionata la ricorrenza verrà richiesta la data di riferimento dell'evento.\\




\subsection{Menù}
L'applicazione offre due menù differenti, il menù delle funzionalità e il menù delle impostazioni.\\
Il menù delle funzionalità si trova nella parte inferiore dello schermo, mentre per accedere al menù delle impostazioni, bisognerà cliccare la relativa icona del menù presente nella toolbar.\\
Interagendo con il menù delle impostazioni laterale si potranno gestire le informazioni personali dell'utente e gestire le informazioni del suo gruppo.\\
Nella pagina riguardante il profilo dell'utente sarà possibile visualizzare le informazioni personali come il nome, l'email e il gruppo a cui esso appartiene, queste informazioni sono modificabili in qualsiasi momento.\\
Nella pagina riguardante il gruppo invece sarà possibile visualizzare le informazioni principali riguardanti il gruppo a cui l'utente appartiene, come: il nome, l'immagine e gli utente che appartengono al gruppo. Le informazioni modificabili in questa pagina sono l'immagine del gruppo, il nome del gruppo e la possibilità di invitare altre persone ad unirsi al gruppo tramite invito, (verrà inviato all'utente un codice di invito che dovrà inserire al momento della registrazione).\\
