\chapter{Applicazione}                %crea il capitolo
%%%%%%%%%%%%%%%%%%%%%%%%%%%%%%%%%%%%%%%%%imposta l'intestazione di pagina
\lhead[\fancyplain{}{\bfseries\thepage}]{\fancyplain{}{\bfseries\rightmark}}
\pagenumbering{arabic}                  %mette i numeri arabi
\section{Sviluppo}                 %crea la sezione

L'applicazione \'e stata sviluppata per risolvere problemi gestionali che vengono riscoontrati da chi vive assieme ad altre persone e deve divedere spese come affitto bollette, beni per la casa ecc

Inizialmene \'e stata definita l'architettura dell'applicazione, le funzionalit\'a principali e una bozza del design.
Lo sviluppo dell'applicazione inizialmente utilizzava Java come linguaggio di programmazione predefinito per la parte Android client, e Firebase come BaS che offriva funzionalit\'a utili per la gestione degli utenti, autenticazione tramite social login e cloud database realtime.
In seguito dopo aver testato le funzionalit\'a di Java e le caratteristiche di firebase furono presi in considerazione Kotlin come linguaggio di programmazione per Android, in alternativa a Java e Firestore come database considerata da Google la versione successiva di Firebase

Il codice quindi \'e stato completamente riscritto utilizzando kotlin e Firestore In modo tale da avere anche un confronto in termini di prestazione complessit\'a del codice linee di codice e sintassi
