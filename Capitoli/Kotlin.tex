\chapter{Kotlin}                %crea il capitolo
%%%%%%%%%%%%%%%%%%%%%%%%%%%%%%%%%%%%%%%%%imposta l'intestazione di pagina
\lhead[\fancyplain{}{\bfseries\thepage}]{\fancyplain{}{\bfseries\rightmark}}
\pagenumbering{arabic}                  %mette i numeri arabi
\section{Definizione}                 %crea la sezione
Kotlin è un linguaggio di programmazione basato sulla JVM ( Java Virtual Machine) di conseguenza il codice risultante alla compilazione sar\`a in Java bytecode.\\
Alcune citazioni \cite{K1,K2}.\\



\subsection{Caratteristiche}
Questa \`e una sezione \cite{K1}, con dei punti:
\begin{description}                     %crea un elenco descrittivo
  \item[Variabili] blabla;
  \item[Funzioni] blabla;
  \item[Classi] blabla;

\end{description}
%%%%%%%%%%%%%%%%%%%%%%%%%%%%%%%%%%%%%%%%%non numera l'ultima pagina sinistra

\clearpage{\pagestyle{empty}\cleardoublepage}
