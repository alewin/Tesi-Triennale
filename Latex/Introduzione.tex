\chapter*{Introduzione}                 %crea l'introduzione (un capitolo
                                        %   non numerato)

%%%%%%%%%%%%%%%%%%%%%%%%%%%%%%%%%%%%%%%%%imposta l'intestazione di pagina
\rhead[\fancyplain{}{\bfseries
Introduzione}]{\fancyplain{}{\bfseries\thepage}}
\lhead[\fancyplain{}{\bfseries\thepage}]{\fancyplain{}{\bfseries
INTRODUZIONE}}
%%%%%%%%%%%%%%%%%%%%%%%%%%%%%%%%%%%%%%%%%aggiunge la voce Introduzione
                                        %   nell'indice
\addcontentsline{toc}{chapter}{Introduzione}
L'idea di realizzare un'applicazione gestionale su piattaforma Android, è nata dall'esigenza di risolvere alcune problematiche reali riscontrate durante l'organizzazione di eventi, spese e commissioni da svolgere fra gruppi di persone.\\
Lo sviluppo iniziale si avvaleva di Java come linguaggio di programmazione, successivamente con l'annuncio ufficiale a Maggio 2017 del supporto di Google a Kotlin, come nuovo linguaggio di programmazione per Android, l'applicazione è stata riscritta completamente, utilizzando Kotlin.\\ % potrebbe presto sostituire Java nello sviluppo mobile Android.
Lo sviluppo dell'applicazione in Kotlin ha permesso di analizzare le funzionalità del nuovo linguaggio sia teoricamente che con un'implementazione pratica che ha reso possibile anche un confronto diretto con Java.\\
In questa tesi verranno illustrate le principali funzionalità e caratteristiche del linguaggio Kotlin, e del BaaS Firebase utilizzato per la gestione della parte server. Dal terzo capitolo invece verrà presa in considerazione l'applicazione, mostrando l'architettura ad alto livello dell'infrastruttura dei servizi server e delle funzionalità dell'applicazione. Successivamente verranno analizzate porzioni di codice rilevanti e discusse le implementazioni sviluppate per la realizzazione della parte client dell'applicazione. Infine nell'ultimo capitolo viene descritto l'attuale stato dell'arte, il confronto fra Java e Kotlin ed eventuali sviluppi futuri.
%%%%%%%%%%%%%%%%%%%%%%%%%%%%%%%%%%%%%%%%%non numera l'ultima pagina sinistra
\clearpage{\pagestyle{empty}\cleardoublepage}
